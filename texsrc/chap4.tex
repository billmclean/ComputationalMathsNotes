\chapter[Finite elements for parabolic problems]{Finite elements for \\
parabolic problems in 1D}

We now consider a more general parabolic PDE with mixed boundary conditions,
\begin{equation}\label{eq: parabolic ivp 1d}
\begin{aligned}
u_t+\mathcal{L}u&=f(x,t)&&\text{for $0<x<L$ and $0<t<T$,}\\
u&=\gamma_0(t)&&\text{at $x=0$, for $0<t<T$,}\\
au&=\gamma_L(t)&&\text{at $x=L$, for $0<t<T$,}\\
u&=u_0(x)&&\text{for $0<x<L$ when $t=0$,}
\end{aligned}
\end{equation}
where $\mathcal{L}u=-\bigl(a(x)u'\bigr)'+c(x)u$ as before 
in~\eqref{eq: L self-adjoint}.  The identity~\eqref{eq: Lu v by parts} implies 
that, for any test function~$v(x)$,
\[
\int_0^Lu_tv\,dx+\int_0^L\bigl(a(x)u_xv_x+c(x)uv\bigr)\,dx
	=\gamma_L(t)v(L)+\int_0^Lf(x,t)v\,dx
\quad\text{provided $v(0)=0$.}
\]

\section{Semidiscrete method}
Let $V_h$ denote the space of continuous, piecewise-linear functions for a 
given mesh~\eqref{eq: 1d nodes} on the spatial interval~$[0,L]$.  We define the 
solution set (now dependent on~$t$) and test space by
\[
S_h(t)=\{\,v\in V_h:v(0)=\gamma_0(t)\,\}
\quad\text{and}\quad
T_h=\{\,v\in V_h:v(0)=0\,\}.
\]
We also choose $u_{0h}\in V_h$ such that $u_0\approx u_{0h}$; the simplest 
choice would be the interpolant $u_{0h}=\mathcal{Q}_1u_0$. The semidiscrete
finite element solution~$u_h(x,t)$ is then defined for~$0\le t\le T$ by 
requiring that $u_h(\cdot,t)\in S_h(t)$ and
\begin{equation}\label{eq: semidiscrete fem 1d}
\int_0^L(u_h)_tv\,dx+\int_0^L\bigl(a(x)u_xv_x+c(x)uv\bigr)\,dx
    =\gamma_L(t)v(L)+\int_0^Lf(x,t)v\,dx
\quad\text{for all $v\in T_h$,}
\end{equation}
with $u_h(0)=u_{0h}$. The stiffness matrix~$\boldsymbol{A}$ and mass 
matrix~$\boldsymbol{C}$ are the same as in the stationary problem 
(section~\ref{sec: self-adjoint 1d}), but the nodal 
values of the solution are now time-dependent, with
\[
u_h(x,t)=\sum_{q=0}^P U_q(t)\chi_q(x)\quad\text{for $0\le x\le L$,}\quad
\text{where $U_q(t)=u_h(x_q,t)$.}
\]
Likewise, the load vector is time-dependent in general, since
\[
f_p(t)=\int_0^Lf(x,t)\chi_p(x)\,dx,
\]
and since
\[
\int_0^L(u_h)_t\chi_p\,dx=\int_0^L\sum_{q=0}^P
    \frac{dU_q}{dt}\,\chi_q(x)\chi_p(x)\,dx=\sum_{q=0}^P m_{pq}\,\frac{dU_q}{dt}
    \quad\text{where $m_{pq}=\int_0^L\chi_q\chi_p\,dx$,}
\]
we see by choosing $v=\chi_p$ in~\eqref{eq: semidiscrete fem 1d} that
\[
\sum_{q=0}^P\biggl(m_{pq}\,\frac{dU_q}{dt}+a_{pq}U_q+c_{pq}U_q\biggr)
    =\gamma_L(t)\chi_p(L)+f_p(t),
\]
with $U_q(0)=u_{0h}(x_q)$.  Since $U_0(t)=u_h(x_0,t)=\gamma_0(t)$,
\[
\sum_{q=1}^P\biggl(m_{pq}\,\frac{dU_q}{dt}+a_{pq}U_k+c_{pq}U_q\biggr)
    =f_p(t)+g_p(t)
\]
where
\[
g_p(t)=\gamma_L(t)\delta_{pP}-m_{p0}\gamma_0'(t)-(a_{p0}+c_{p0})\gamma_0(t),
\]
and so the nodal values satisfy the $P\times P$ system of linear ODEs,
\begin{equation}\label{eq: semidiscrete fem 1d ODE}
\boldsymbol{M}\,\frac{d\boldsymbol{U}}{dt}
+(\boldsymbol{A}+\boldsymbol{C})\boldsymbol{U}=\boldsymbol{f}(t)
    +\boldsymbol{g}(t)\quad\text{for $0\le t\le T$,}
    \quad\text{with $\boldsymbol{U}(0)=\boldsymbol{U}_0$,}
\end{equation}
where 
\[
\boldsymbol{M}=[m_{pq}]_{p,q=1}^P,\qquad 
\boldsymbol{A}=[a_{pq}]_{p,q=1}^P,\qquad
\boldsymbol{C}=[c_{pq}]_{p,q=1}^P,
\]
and
\[
\boldsymbol{f}(t)=[f_p(t)]_{p=1}^P,\qquad
\boldsymbol{g}(t)=[g_p(t)]_{p=1}^P,\qquad
\boldsymbol{U}_0=[u_{0h}(x_p)]_{p=1}^P.
\]

\section{Time stepping}\label{sec: time stepping}
We can use finite differences for the time integration, putting
\[
t_n=n\,\Delta t\quad\text{for $0\le n\le N$,}
    \quad\text{where $\Delta t=\frac{T}{N}$,}
\]
and seeking $U^n_p\approx u_h(x_p,t_n)\approx u(x_p,t_n)$.

\begin{example}
The \emph{forward Euler method} based on~\eqref{eq: semidiscrete fem 1d ODE} is
\[
\boldsymbol{M}\,\frac{\boldsymbol{U}^{n+1}-\boldsymbol{U}^n}{\Delta t}
+(\boldsymbol{A}+\boldsymbol{C})\boldsymbol{U}^n=\boldsymbol{f}^n
    +\boldsymbol{g}^n\quad\text{for $0\le n\le N-1$,}
    \quad\text{with $\boldsymbol{U}^0=\boldsymbol{U}_0$,}
\]
where $\boldsymbol{U}^n=[U^n_p]_{p=1}^P$, 
$\boldsymbol{f}^n=\boldsymbol{f}(t_n)$~and 
$\boldsymbol{g}^n=\boldsymbol{g}(t_n)$.  Notice that this method is not 
actually explicit, due to the presence of the matrix~$\boldsymbol{M}$: at the 
$n$th time step we have to solve the linear system
\[
\boldsymbol{M}\boldsymbol{U}^{n+1}=\bigl(\boldsymbol{M}
-\Delta t\,(\boldsymbol{A}+\boldsymbol{C})\bigr)\boldsymbol{U}^n
    +\Delta t\,\bigl(\boldsymbol{f}^n+\boldsymbol{g}^n\bigr).
\]

\end{example}

\begin{example}
The \emph{backward Euler method} based on~\eqref{eq: semidiscrete fem 1d ODE} is
\[
\boldsymbol{M}\,\frac{\boldsymbol{U}^{n}-\boldsymbol{U}^{n-1}}{\Delta t}
+(\boldsymbol{A}+\boldsymbol{C})\boldsymbol{U}^n=\boldsymbol{f}^n
    +\boldsymbol{g}^n\quad\text{for $1\le n\le N$,}
    \quad\text{with $\boldsymbol{U}^0=\boldsymbol{U}_0$,}
\]
which requires that we solve the linear systems
\[
\bigl(\boldsymbol{M}+\Delta t\,(\boldsymbol{A}+\boldsymbol{C})\bigr) 
    \boldsymbol{U}^n=\boldsymbol{M}\boldsymbol{U}^{n-1}
    +\Delta t\,\bigl(\boldsymbol{f}^n+\boldsymbol{g}^n\bigr).
\]
\end{example}

\begin{example}
The \emph{Crank--Nicolson method} based on~\eqref{eq: semidiscrete fem 1d ODE} 
is
\[
\boldsymbol{M}\,\frac{\boldsymbol{U}^{n}-\boldsymbol{U}^{n-1}}{\Delta t}
+(\boldsymbol{A}+\boldsymbol{C})\boldsymbol{U}^{n-1/2}=\boldsymbol{f}^{n-1/2}
    +\boldsymbol{g}^{n-1/2}\quad\text{for $1\le n\le N$,}
    \quad\text{with $\boldsymbol{U}^0=\boldsymbol{U}_0$,}
\]
which requires that we solve the linear systems
\[
\bigl(\boldsymbol{M}+\tfrac12\Delta t\,(\boldsymbol{A}+\boldsymbol{C})\bigr) 
    \boldsymbol{U}^n
=\bigl(\boldsymbol{M}-\tfrac12\Delta t\,(\boldsymbol{A}+\boldsymbol{C})\bigr) 
    \boldsymbol{U}^{n-1}
    +\Delta t\,\bigl(\boldsymbol{f}^{n-1/2}+\boldsymbol{g}^{n-1/2}\bigr).
\]
\end{example}

\section{Discrete separation of variables}
Each of the matrices $\boldsymbol{A}$, $\boldsymbol{C}$~and $\boldsymbol{M}$ is 
real and symmetric, and in addition $\boldsymbol{M}$ is positive-definite, so 
there exist generalised eigenpairs $(\boldsymbol{\Phi}_k,\Lambda_k)$ such that 
\[
(\boldsymbol{A}+\boldsymbol{C})\boldsymbol{\Phi}_k
    =\Lambda_k\boldsymbol{M}\boldsymbol{\Phi}_k,\qquad
\boldsymbol{\Phi}_j^T(\boldsymbol{A}+\boldsymbol{C})\boldsymbol{\Phi}_k
    =\Lambda_k\delta_{jk},\qquad
\boldsymbol{\Phi}_j^T\boldsymbol{M}\boldsymbol{\Phi}_k=\delta_{jk}
\]
for $j$, $k\in\{1,2,\ldots,P\}$.

We will consider the special case when $f(x,t)\equiv0$, 
$\gamma_0(t)\equiv0$~and $\gamma_L(t)\equiv0$, for the special choice of 
discrete initial data
\[
\boldsymbol{U}_0=\Phi_k.
\]
It follows that the semidiscrete initial-value 
problem~\eqref{eq: semidiscrete fem 1d ODE} simplifies to
\[
\boldsymbol{M}\,\frac{d\boldsymbol{U}}{dt}
+(\boldsymbol{A}+\boldsymbol{C})\boldsymbol{U}=\boldsymbol{0}
    \quad\text{for $0\le t\le T$,}
    \quad\text{with $\boldsymbol{U}(0)=\boldsymbol{\Phi}_k$,}
\]
which has the solution
\begin{equation}
\boldsymbol{U}(t)=e^{-\Lambda_kt}\boldsymbol{\Phi}_k.
\end{equation}
Let us compare this continuous-time solution with the discrete-time solutions 
for the three time-stepping schemes described in 
section~\ref{sec: time stepping}.

\begin{example}
The forward Euler method in this special case is
\[
\boldsymbol{M}\boldsymbol{U}^{n+1}=\bigl(\boldsymbol{M}
-\Delta t\,(\boldsymbol{A}+\boldsymbol{C})\bigr)\boldsymbol{U}^n
\]
and since
\[
\bigl(\boldsymbol{M} 
-\Delta t\,(\boldsymbol{A}+\boldsymbol{C})\bigr)\boldsymbol{\Phi}_k
    =\boldsymbol{M}\Phi_k-\Delta t\,\Lambda_k\boldsymbol{M}\boldsymbol{\Phi}_k
    =\boldsymbol{M}(1-\Delta t\,\Lambda_k)\boldsymbol{\Phi}_k
\]
we see by induction on $n$ that
\[
\boldsymbol{U}^n=(1-\Delta t\,\Lambda_k)^n\boldsymbol{\Phi}_k.
\]
\end{example}



