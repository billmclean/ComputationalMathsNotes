\chapter[Finite elements in 1D]{Finite elements for \\
stationary problems in 1D}\label{chap: FEM 1d}

Recall our model two-point boundary-value problem~\ref{eq: model 1d}, 
\begin{equation}\label{eq: model 1d chap 2}
-u''=f(x)\quad\text{for $0<x<L$,}
	\quad\text{with $u(0)=\gamma_0$ and $u(L)=\gamma_L$.}
\end{equation}
Multiply both sides of the ODE by a \emph{test function}~$v$ and integrate to 
obtain
\[
-\int_0^L u''(x)v(x)\,dx=\int_0^L f(x)v(x)\,dx.
\]
Provided $v'$ exists and is continuous on~$[0,L]$, we can integrate by parts to 
obtain
\begin{equation}\label{eq: int by parts}
-\int_0^L u''(x)v(x)\,dx=-\bigl[u'(x)v(x)\bigr]_0^L+\int_0^Lu'(x)v'(x)\,dx
\end{equation}
and therefore
\begin{equation}\label{eq: model 1d weak}
\int_0^L u'(x)v'(x)\,dx=\int_0^L f(x)v(x)\,dx
	\quad\text{provided $v(0)=0=v(L)$.}
\end{equation}
This observation is the basis of the finite element method.

\section{Piecewise-linear functions}
Suppose that we partition the interval~$[0,L]$ into $P$~subintervals, not 
necessarily of equal length, by choosing grid points, or \emph{nodes},
\begin{equation}\label{eq: 1d nodes}
0=x_0<x_1<x_2<\cdots<x_P=L.
\end{equation}
Denote the length of the $p$th subinterval, or \emph{element}, 
by~$h_p=x_p-x_{p-1}$, for~$1\le p\le P$, and denote the maximum width 
by~$h=\max_{1\le p\le P}h_p$.  A function $v:[0,L]\to\mathbb{R}$ is 
\emph{piecewise-linear} (with respect to the chosen grid points~$x_p$) if 
there exist polynomials $v_1$, $v_2$, \dots, $v_P$, each of degree at most~$1$, 
such that
\begin{equation}\label{eq: v piecewise}
v(x)=v_p(x)\quad\text{for $x_{p-1}<x<x_p$ and $1\le p\le P$.}
\end{equation}
Such a function~$v$ is continuous on~$[0,L]$ if and only if the $P$~polynomials 
satisfy
\begin{equation}\label{eq: v cts}
v_p(x_p)=v_{p+1}(x_p)\quad\text{for $1\le p\le P-1$.}
\end{equation}
The set of all continuous, piecewise-linear functions forms a vector 
space~$V_h$.  Each $v\in V_h$ is determined by its values at the grid points, 
since
\[
v(x)=v_p(x)=\frac{1}{h_p}\bigl((x_p-x)v(x_{p-1})+(x-x_{p-1})v(x_p)\bigr)
    \quad\text{for $x_{p-1}\le x\le x_p$.}
\]
In particular, the $q$th \emph{nodal basis function} $\chi_q\in V_h$ is the 
unique continuous, piecewise-linear function satisfying
\begin{equation}\label{eq: chi q x p}
\chi_q(x_p)=\delta_{pq}\quad\text{for $p$, $q\in\{0, 1, 2, \ldots, P\}$,}
\end{equation}
where $\delta_{pq}$ is the Kronecker delta, that is,
\[
\delta_{pq}=\begin{cases}
1,&\text{if $p=q$,}\\ 0,&\text{if $p\ne q$.} 
\end{cases}
\]
The property~\eqref{eq: chi p x q} implies that each~$v\in V_h$ has the 
representation
\[
v(x)=\sum_{q=0}^P v(x_q)\chi_q(x)\quad\text{for $0\le x\le L$,}
\]
since the sum on the right defines a continuous, piecewise-linear function that 
equals $v(x_p)$ when~$x=x_p$, for~$0\le p\le M$.  It follows that 
$\{\chi_q\}_{q=0}^P$ is a basis for~$V_h$, and therefore that $\dim V_h=P+1$.

In the finite element method for our model 2-point boundary-value 
problem~\eqref{eq: model 1d chap 2}, we define the \emph{solution set} (or
\emph{trial set})
\[
S_h=\{\,v\in V_h:\text{$v(0)=\gamma_0$ and $v(L)=\gamma_L$}\,\}
\]
and the \emph{test space}
\[
T_h=\{\,v\in V_h:\text{$v(0)=0$ and $v(L)=0$}\,\}.
\]
Notice that $T_h$ is subspace of~$V_h$ with $\dim T_h=P-1$.  Based 
on~\eqref{eq: model 1d weak}, the \emph{finite element solution}~$u_h$
of~\eqref{eq: model 1d chap 2} is determined by requiring that
\[
u_h\in S_h\qquad\text{and}\qquad
\int_0^L u_h'(x)v'(x)\,dx=\int_0^L f(x)v(x)\,dx \quad\text{for all $v\in T_h$.}
\]
Since $\{\chi_p\}_{p=1}^{P-1}$ is a basis for~$T_h$, and since the equations 
are linear in~$v$, it suffices that $u_h$ satisfies
\begin{equation}\label{eq: model FEM chi p}
\int_0^L u_h'(x)\chi_p'(x)\,dx=\int_0^L f(x)\chi_p(x)\,dx 
    \quad\text{for $1\le p\le P-1$.}
\end{equation}
Moreover, 
\begin{equation}\label{eq: uh U 1d}
u_h(x)=\sum_{q=0}^P U_q\chi_q(x)\quad\text{for $0\le x\le L$,}\quad
\text{where $U_q=u_h(x_q)$,}
\end{equation}
so
\[
\int_0^Lu_h'(x)\chi_p'(x)\,dx
    =\int_0^L\biggl(\sum_{k=0}^PU_q\chi_q'(x)\biggr)\chi_p'(x)\,dx
    =\sum_{q=0}^P\biggl(\int_0^L\chi_q'(x)\chi_p'(x)\,dx\biggr)U_q.
\]
Therefore, by letting
\[
a_{pq}=\int_0^L\chi_q'(x)\chi_p'(x)\,dx
\quad\text{and}\quad
f_p=\int_0^Lf(x)\chi_p(x)\,dx,
\]
we can write \eqref{eq: model FEM chi p} as
\[
\sum_{q=0}^P a_{pq}U_q=f_p\quad\text{for $1\le p\le P-1$.}
\]
Since $u_h\in S_h$, we have $U_0=u_h(x_0)=u_h(0)=\gamma_0$~and 
$U_P=u_h(x_P)=u_h(L)=\gamma_L$, leading to a $(P-1)\times(P-1)$ linear system 
for the unknowns $U_1$, $U_2$, \dots, $U_{P-1}$, namely
\begin{equation}\label{eq: model FEM eqns}
\sum_{q=1}^{P-1}a_{pq}U_q=f_p-a_{p0}\gamma_0-a_{pM}\gamma_L
    \quad\text{for $1\le p\le P-1$.}
\end{equation}
The coefficients on the left-hand side of~\eqref{eq: model FEM eqns} form the
\emph{stiffness matrix}~$\boldsymbol{A}$, which has the following properties.

\begin{theorem}
If $|p-q|\ge2$ then $a_{pq}=0$.  The diagonal values are
\[
a_{00}=\frac{1}{h_1},\qquad
\text{$a_{pp}=\frac{1}{h_p}+\frac{1}{h_{p+1}}$ for $1\le p\le P-1$,}\qquad
a_{PP}=\frac{1}{h_P},
\]
and the off-diagonal values are
\[
a_{p-1,p}=a_{p,p-1}=\frac{-1}{h_p}\quad\text{for $1\le p\le P$.}
\]
Furthermore, the $(P-1)\times(P-1)$ symmetric, tridiagonal 
matrix~$\boldsymbol{A}=[a_{pq}]_{p,q=1}^{P-1}$ is strictly positive-definite.
\end{theorem}
\begin{proof}
It is easy to see that for~$1\le p\le P-1$,
\[
\chi_p'(x)=\begin{cases}
1/h_p,&x_{p-1}<x<x_p,\\
-1/h_{p+1},&x_p<x<x_{p+1},\\
0,&\text{otherwise,}
\end{cases}
\]
with the first case missing when~$p=0$, and the second when~$p=P$.  Thus,
the supports of $\chi_p'$ and $\chi_q'$ overlap iff $|p-q|\le1$; otherwise,
if $|p-q|\ge2$, then $a_{pq}=0$. The diagonal values are
\[
a_{00}=\int_0^L\chi_0'(x)^2\,dx
    =\int_{x_0}^{x_1}\biggl(\frac{1}{h_1}\biggr)^2\,dx
    =\frac{x_1-x_0}{h_1^2}=\frac{1}{h_1}
\]
and
\[
a_{PP}=\int_0^L\chi'_P(x)^2\,dx
    =\int_{x_{P-1}}^{x_P}\biggl(\frac{-1}{h_P}\biggr)^2\,dx
    =\frac{x_P-x_{P-1}}{h_P^2}=\frac{1}{h_P},
\]
with
\[
a_{pp}=\int_{x_{p-1}}^{x_p}\biggl(\frac{1}{h_p}\biggr)^2\,dx
      +\int_{x_p}^{x_{p+1}}\biggl(\frac{-1}{h_{p+1}}\biggr)^2\,dx
        =\frac{1}{h_p}+\frac{1}{h_{p+1}}
\]
for $1\le p\le P-1$.  The off-diagonal values are
\[
a_{p-1,p}=\int_0^L\chi'_p(x)\chi'_{p-1}(x)\,dx
    =\int_{x_{p-1}}^{x_p}\biggl(\frac{-1}{h_p}\biggr)
    \biggl(\frac{1}{h_p}\biggr)\,dx=\frac{-(x_p-x_{p-1})}{h_p^2}=\frac{-1}{h_p}
\]
for $1\le p\le P$. 

The matrix~$\boldsymbol{A}$ is thus tridiagonal and symmetric.  Given
$\boldsymbol{V}=[V_q]_{q=1}^{P-1}\in\mathbb{R}^{P-1}$
we define $v\in T_h$ by $v(x)=\sum_{q=1}^{P-1}V_q\chi_q(x)$ and observe that
\begin{align*}
\boldsymbol{V}^T\boldsymbol{A}\boldsymbol{V}
    &=\sum_{p=1}^{P-1}\sum_{q=1}^{P-1}V_pa_{pq}V_q
    =\sum_{p=1}^{P-1}\sum_{q=1}^{P-1}V_pV_q\int_0^L\chi_q'(x)\chi_p'(x)\,dx\\
    &=\int_0^L\biggl(\sum_{q=0}^{P-1}V_q\chi_q'(x)\biggr)
             \biggl(\sum_{p=0}^{P-1}V_p\chi_p'(x)\biggr)\,dx
    =\int_0^L\bigl(v'(x)\bigr)^2\,dx\ge0.
\end{align*}
Thus, $\boldsymbol{A}$ is positive-semidefinite.  To see that
$\boldsymbol{A}$ is in fact \emph{strictly} positive-definite, suppose that
$\boldsymbol{V}^T\boldsymbol{A}\boldsymbol{V}=0$. Then 
$\int_0^L\bigl(v'(x)\bigr)^2\,dx=0$ so $v'=0$ and thus $v$ is constant 
on~$[0,L]$. Since $v(0)=0=v(L)$, the function~$v$ must be identically zero, 
and hence $V_q=0$ for~$1\le q\le M-1$.   That is, 
$\boldsymbol{V}^T\boldsymbol{A}\boldsymbol{V}=0$ implies 
$\boldsymbol{V}=\boldsymbol{0}$.
\end{proof}

For example, if $P=6$ then the $5\times 5$ system of 
equations~\eqref{eq: model FEM eqns} can be written in matrix form as
\[
\begin{bmatrix}
 h_1^{-1}+h_2^{-1}&        -h_2^{-1}&&&\\
         -h_2^{-1}&h_2^{-1}+h_3^{-1}&-h_3^{-1}&&\\
        &-h_3^{-1}&h_3^{-1}+h_4^{-1}&-h_4^{-1}&\\
       &&-h_4^{-1}&h_4^{-1}+h_5^{-1}&-h_5^{-1}\\
      &&&-h_5^{-1}&h_5^{-1}+h_6^{-1}\\
\end{bmatrix}
\begin{bmatrix}U_1\\ U_2\\ U_3\\ U_4\\ U_5\end{bmatrix}
=\begin{bmatrix}f_1\\ f_2\\ f_3\\ f_4\\ f_5\end{bmatrix}
+\begin{bmatrix}h_1^{-1}\gamma_0\\ \\ \\ \\ h_6^{-1}\gamma_L\end{bmatrix}.
\]
Being positive-definite, the stiffness matrix is non-singular so this linear 
system has a unique solution, which can be computed using the algorithms 
described in section~\ref{sec: sym tridiagonal}.

\section{General self-adjoint problems}\label{sec: self-adjoint 1d}
A second-order linear differential operator~$\mathcal{L}$ is 
\emph{formally self-adjoint} if it can be written in the form
\begin{equation}\label{eq: L self-adjoint}
(\mathcal{L}u)(x)=-\bigl(a(x)u'\bigr)'+c(x)u(x).
\end{equation}
Here, the coefficients $a(x)$ and $c(x)$ are assumed to have the same 
properties as in Chapter~\ref{chap: finite diff 1d}; in particular, $a(x)$ must 
satisfy the lower bound~\eqref{eq: ellipticity 1d}. For such an $\mathcal{L}$, 
consider the following two-point boundary-value problem with \emph{mixed 
boundary conditions},
\begin{equation}\label{eq: self-adjoint mixed}
\mathcal{L}u=f(x)\quad\text{for $0<x<L$,}\quad
\text{with $u(0)=\gamma_0$ and $a(L)u'(L)=\gamma_L$.}
\end{equation}
Integration by parts implies that
\begin{equation}\label{eq: Lu v by parts}
\int_0^L(\mathcal{L}u)(x)v(x)\,dx
    =-\bigl[a(x)u'(x)v(x)\bigr]_0^L+\int_0^L\bigl(a(x)u'(x)v'(x)+c(x)u(x)v(x)
        \bigr)\,dx,
\end{equation}
so any solution~$u$ of~\eqref{eq: self-adjoint mixed} must satisfy
\begin{equation}\label{eq: Lu=f weak 1d}
\int_0^L\bigl(a(x)u'(x)v'(x)+c(x)u(x)v(x)\bigr)\,dx
    =\gamma_Lv(L)+\int_0^Lf(x)v(x)\,dx
    \quad\text{provided $v(0)=0$.}
\end{equation}
Given nodes~\eqref{eq: 1d nodes}, we therefore define the solution set~$S_h$ and 
test space~$T_h$ by
\[
S_h=\{\,v\in V_h:v(0)=\gamma_0\,\}
\quad\text{and}\quad
T_h=\{\,v\in V_h:v(0)=0\,\},
\]
and require that the finite element solution~$u_h\in S_h$ satisfy
\begin{equation}\label{eq: self-adjoint mixed bc FEM}
\int_0^L\bigl(a(x)u_h'(x)v'(x)+c(x)u_h(x)v(x)\bigr)\,dx
    =\gamma_Lv(L)+\int_0^Lf(x)v(x)\,dx
    \quad\text{for all $v\in T_h$.}
\end{equation}
Equivalently, since $\{\chi_j\}_{j=1}^M$ is a basis for~$T_h$, we require
\[
\int_0^L\bigl(a(x)u_h'(x)\chi_j'(x)+c(x)u_h(x)\chi_j(x)\bigr)\,dx
    =\gamma_L\chi_j(L)+\int_0^Lf(x)\chi_j(x)\,dx
    \quad\text{for $1\le j\le M$.}
\]
Inserting the representation~\eqref{eq: uh U 1d} yields the system of linear 
equations
\[
\sum_{q=0}^P\bigl(a_{pq}U_q+c_{pq}U_q\bigr)=\gamma_L\chi_p(L)+f_p
    \quad\text{for $1\le p\le P$,}
\]
where
\[
a_{pq}=\int_0^La(x)\chi_q'(x)\chi_p'(x)\,dx,\quad
c_{pq}=\int_0^Lc(x)\chi_q(x)\chi_p(x)\,dx,\quad
f_p=\int_0^Lf(x)\chi_p(x)\,dx.
\]
Moving $U_0=\gamma_0$ to the right-hand side leads to a $P\times P$ linear 
system,
\[
\sum_{q=1}^P\bigl(a_{pq}+c_{pq})U_q
    =\gamma_L\chi_p(L)+f_p-(a_{p0}+c_{p0})\gamma_0
    \quad\text{for $1\le p\le P$,}
\]
or, in matrix notation,
\begin{equation}\label{eq: self-adjoint mixed equations}
\bigl(\boldsymbol{A}+\boldsymbol{C}\bigr)\boldsymbol{U}
    =\boldsymbol{f}+\boldsymbol{g},
\end{equation}
where $\boldsymbol{A}=[a_{pq}]_{p,q=1}^P$, $\boldsymbol{C}=[c_{pq}]_{p,q=1}^P$,
$\boldsymbol{f}=[f_p]_{p=1}^P$ and $\boldsymbol{g}=[g_p]_{p=1}^P$, where
\[
g_1=-(a_{10}+c_{10})\gamma_0,\qquad
\text{$g_p=0$ for $2\le p\le M-1$,}\qquad
g_P=\gamma_L.
\]
We again refer to~$\boldsymbol{A}$ as the stiffness matrix, whereas 
$\boldsymbol{C}$ is called the \emph{mass matrix}.  On the right-hand side, 
$\boldsymbol{f}$ is called the \emph{load vector}.  This terminology reflects 
the historical origins of finite element methods in structural engineering.

\section{Matrix assembly element-by-element}\label{sec: matrix assembly 1d}

In the previous section, we used the nodal basis functions~$\chi_q$ to set up 
the linear system~\eqref{eq: self-adjoint mixed equations}, but this approach 
becomes very complicated in 2D~or 3D, or even in 1D with higher-order elements.
Instead, a simpler method is to assemble the matrices $\boldsymbol{A}$~and 
$\boldsymbol{C}$, and the vector~$\boldsymbol{f}$, element-by-element.  
For~$1\le p\le P$, we put $\mathsf{n}^\brak{p}_1=x_{p-1}$~and
$\mathsf{n}^\brak{p}_2=x_p$ so that the $p$th 
element is $[x_{p-1},x_p]=[\mathsf{n}^\brak{p}_1,\mathsf{n}^\brak{p}_2]$.  The
\emph{linear shape functions} for this element are defined by
\[
\psi^\brak{p}_1(x)=\frac{x_p-x}{h_p}
\quad\text{and}\quad
\psi^\brak{p}_2(x)=\frac{x-x_{p-1}}{h_p}
\quad\text{for $x_{p-1}\le x\le x_p$,}
\]
and satisfy
\[
\psi^\brak{p}_j(\mathsf{n}^\brak{p}_k)=\delta_{jk}
    \quad\text{for $j$, $k\in\{1,2\}$,}
\]
so that for any~$v\in V_h$,
\[
v(x)=v(\mathsf{n}^\brak{p}_1)\psi^\brak{p}_1(x)
    +v(\mathsf{n}^\brak{p}_2)\psi^\brak{p}_2(x)
    \quad\text{for $x\in[\mathsf{n}^\brak{p}_1,\mathsf{n}^\brak{p}_2]$.}
\]
The \emph{element stiffness matrix} is defined by
\[
\boldsymbol{A}^\brak{p}=\begin{bmatrix}
a^\brak{p}_{11}&a^\brak{p}_{12}\\
a^\brak{p}_{21}&a^\brak{p}_{22}\end{bmatrix}
\quad\text{where}\quad
a^\brak{p}_{jk}=\int_{x_{p-1}}^{x_p}a(x)\bigl(\psi_k^\brak{p}\bigr)'(x)
    \bigl(\psi_j^\brak{p}\bigr)'(x)\,dx,
\]
the \emph{element mass matrix} by
\[
\boldsymbol{C}^\brak{p}=\begin{bmatrix}
c^\brak{p}_{11}&c^\brak{p}_{12}\\
c^\brak{p}_{21}&c^\brak{p}_{22}\end{bmatrix}
\quad\text{where}\quad
c^\brak{p}_{jk}=\int_{x_{p-1}}^{x_p}c(x)\psi_k^\brak{p}(x) 
    \psi_j^\brak{p}(x)\,dx,
\]
and the \emph{element load vector} by
\[
\boldsymbol{f}^\brak{p}=\begin{bmatrix}f^\brak{p}_1\\ f^\brak{p}_2\end{bmatrix}
\quad\text{where}\quad
f^\brak{p}_j=\int_{x_{p-1}}^{x_p}f(x)\psi_j^\brak{p}(x)\,dx.
\]
We also enumerate the nodes of the mesh so that the \emph{free nodes precede 
the fixed nodes}, where the latter are those at which the value of the 
solution is fixed by a Dirichlet boundary condition.  For our 
problem~\eqref{eq: self-adjoint mixed}, the only fixed node is~$x_0$, so we put
\[
\mathsf{n}_p=x_p\quad\text{for $1\le p\le P$,}
\quad\text{and}\quad\mathsf{n}_{P+1}=x_0.
\]
The $2\times(P+1)$ \emph{connectivity matrix} $\boldsymbol{T}=[t_{jp}]$ is 
defined by
\begin{equation}\label{eq: e pm def}
t_{jp}=r\quad\text{iff}\quad\mathsf{n}^\brak{p}_j=\mathsf{n}_r;
\end{equation}
for example, if $P=6$ then
\[
\boldsymbol{T}=\begin{bmatrix}
7&1&2&3&4&5&6\\
1&2&3&4&5&6&7\end{bmatrix}.
\]
Writing $U^\brak{p}_k=u_h(\mathsf{n}^\brak{p}_k)$~and
$V^\brak{p}_j=v(\mathsf{n}^\brak{p}_j)$, we see that
\[
u_h(x)=\sum_{k=1}^2U^\brak{p}_k\psi^\brak{p}_k(x)
\quad\text{and}\quad
v(x)=\sum_{j=1}^2V^\brak{p}_j\psi^\brak{p}_j(x)
\quad\text{for $x\in[x_{p-1},x_p]$.}
\]
Thus,
\[
\int_0^Lf(x)v(x)\,dx=\sum_{p=1}^P\int_{x_{p-1}}^{x_p}f(x)\sum_{j=1}^2
    V^\brak{p}_j\psi^\brak{p}_j(x)\,dx
    =\sum_{p=1}^P\sum_{j=1}^2 f^\brak{p}_jV_j^\brak{p}
\]
and
\begin{align*}
\int_0^La(x)u_h'(x)v'(x)\,dx
    &=\sum_{p=1}^P\int_{x_{p-1}}^{x_p}a(x)
    \biggl(\sum_{k=1}^2 U^\brak{p}_k(\psi_k^\brak{p})'(s)\biggr)
    \biggl(\sum_{j=1}^2 V^\brak{p}_j(\psi_j^\brak{p})'(x)\biggr)\,dx\\
    &=\sum_{p=1}^P\sum_{k=1}^2\sum_{j=1}^2
    U^\brak{p}_ka^\brak{p}_{jk}V^\brak{p}_j;
\end{align*}
likewise
\[
\int_0^Lc(x)u_h(x)v(x)\,dx=\sum_{p=1}^P\sum_{k=1}^2\sum_{j=1}^2
    U^\brak{p}_kc^\brak{p}_{jk}V^\brak{p}_j
\quad\text{and}\quad
v(L)=V^\brak{P}_2.
\]
Therefore, \eqref{eq: self-adjoint mixed bc FEM} holds iff
\[
\sum_{p=1}^P\sum_{j=1}^2V^\brak{p}_j
\sum_{j=1}^2\bigl(a^\brak{p}_{jk}+c^\brak{p}_{jk} \bigr)U^\brak{p}_k
    =\gamma_LV^\brak{P}_2+\sum_{p=1}^P\sum_{j=1}^2V^\brak{p}_jf^\brak{p}_j
    \quad\text{for all $v\in T_h$.}
\]

Let $\mathfrak{I}_r=\{\,(p,j):\mathsf{n}^\brak{p}_j=\mathsf{n}_r\,\}$, and
define the $P\times(P+1)$ matrices $\boldsymbol{A}=[a_{pq}]$~and 
$\boldsymbol{C}=[c_{pq}]$, and the $P$-dimensional vector~$\boldsymbol{f}$, by
\[
a_{rs}=\sum_{(p,j)\in\mathfrak{I}_r}
    \sum_{(p,k)\in\mathfrak{I}_s}a^\brak{p}_{jk},\qquad
c_{rs}=\sum_{(p,j)\in\mathfrak{I}_r}
    \sum_{(p,k)\in\mathfrak{I}_r}c^\brak{p}_{jk},\qquad
f_r=\sum_{(p,j)\in\mathfrak{I}_r}f^\brak{p}_j,
\]
for $1\le p\le P$ and $1\le q\le P+1$.  Since 
$\mathsf{n}\brak{1}_1=\mathsf{n}_{P+1}=x_0=0$ we have $V^\brak{1}_1=V_{P+1}=0$, 
so
\[
\boldsymbol{V}^T\bigl(\boldsymbol{A}+\boldsymbol{C}\bigr)\boldsymbol{U}
    =\gamma_LV_P+\boldsymbol{V}^T\boldsymbol{f}
    \quad\text{for all $\boldsymbol{V}\in\mathbb{R}^P$,}
\]
and therefore
\begin{equation}\label{eq: A C f example}
\bigl(\boldsymbol{A}+\boldsymbol{C}\bigr)\boldsymbol{U}
    =\gamma_L\boldsymbol{e}_P+\boldsymbol{f}.
\end{equation}
Now partition the matrices as 
$\boldsymbol{A}=[\boldsymbol{A}\free\quad\boldsymbol{A}\fix]$~and
$\boldsymbol{C}=[\boldsymbol{C}\free\quad\boldsymbol{C}\fix]$, where 
$\boldsymbol{A}\free$~and $\boldsymbol{C}\free$ are $P\times P$, and so
$\boldsymbol{A}\fix$~and $\boldsymbol{C}\fix$ are $P\times1$.  Likewise 
partition the vector~$\boldsymbol{U}=[\boldsymbol{U}\free\quad U\fix]^T$ where 
$\boldsymbol{U}\free$ is $P\times1$ and so $U\fix$ is $1\times1$, that is, a 
scalar.  Since $U\fix=U_{P+1}=u_h(\mathsf{n}_{P+1})=u_h(0)=\gamma_0$, 
\[
\bigl(\boldsymbol{A}+\boldsymbol{C}\bigr)\boldsymbol{U}
    =\bigl[(\boldsymbol{A}\free+\boldsymbol{C}\free)\quad
    (\boldsymbol{A}\fix+\boldsymbol{C}\fix)\bigr]
    \begin{bmatrix}\boldsymbol{U}\free\\ U\fix\end{bmatrix}
    =(\boldsymbol{A}\free+\boldsymbol{C}\free)\boldsymbol{U}\free
    +\gamma_0(\boldsymbol{A}\fix+\boldsymbol{C}\fix),
\]
and we conclude that
\[
\bigl(\boldsymbol{A}\free+\boldsymbol{C}\free\bigr)\boldsymbol{U}\free
=\boldsymbol{f}-\gamma_0(\boldsymbol{A}\fix+\boldsymbol{C}\fix)
    +\gamma_L\boldsymbol{e}_P.
\]
Algorithm~\ref{alg: assemble f piecewise linear} shows how, using the 
connectivity matrix~$\boldsymbol{T}=[t_{pm}]$, the global load 
vector~$\boldsymbol{f}$ can be assembled from the element load vectors.  
Likewise, algorithm~\ref{alg: assemble A piecewise linear} shows how the global 
stiffness matrix~$\boldsymbol{A}$ can be assembled from the element stiffness 
matrices.  The global mass matrix~$\boldsymbol{C}$ is assembled in the same way.
In practice, $\boldsymbol{A}$~and $\boldsymbol{C}$ are constructed as 
\emph{sparse arrays} with an appropriate data structure.

\begin{algorithm}
\caption{Assemble the load vector $\boldsymbol{f}$
from~\eqref{eq: A C f example}.}
\label{alg: assemble f piecewise linear}
\begin{algorithmic}
\State Allocate storage for $\boldsymbol{f}=[f_p]\in\mathbb{R}^P$.
\For{$p=1:P$}
    \State $f_p=0$
\EndFor
\For{$p=1:P$}
    \State Compute $\boldsymbol{f}^\brak{p}$
    \For{$j=1:2$}
        \State $r=t_{pj}$\Comment{$\mathsf{n}_r=\mathsf{n}^\brak{p}_j$}
        \If{$r\le M$}
            \State $f_r\gets f_r+f^\brak{p}_j$
        \EndIf
    \EndFor
\EndFor
\end{algorithmic}
\end{algorithm}

\begin{algorithm}
\caption{Assemble the stiffness matrix $\boldsymbol{A}$
from~\eqref{eq: A C f example}.}
\label{alg: assemble A piecewise linear}
\begin{algorithmic}
\State Allocate storage for 
$\boldsymbol{A}=[a_{pq}]\in\mathbb{R}^{P\times(P+1)}$ 
\For{$p=1:P$}
    \For{$q=1:P+1$}
        \State $a_{pq}=0$ 
    \EndFor
\EndFor
\For{$p=1:P$}
    \State Compute $\boldsymbol{A}^\brak{p}$ 
    \For{$j=1:2$}
        \State $r=e_{jp}$
        \If{$r\le P$}
            \For{$k=1:2$}
                \State $s=t_{kp}$
                \State $a_{rs}\gets a_{rs}+a^\brak{p}_{jk}$
            \EndFor
        \EndIf
    \EndFor
\EndFor
\end{algorithmic}
\end{algorithm}

\begin{example}
We will show that for the constant coefficients $a(x)=1$ and $c(x)=1$, the 
element stiffness and mass matrices are simply
\[
\boldsymbol{A}^\brak{p}=\frac{1}{h_p}\begin{bmatrix}1&-1\\ -1&1\end{bmatrix}
\quad\text{and}\quad
\boldsymbol{C}^\brak{p}=\frac{h_p}{6}\begin{bmatrix}2&1\\ 1&2\end{bmatrix}.
\]
In fact, since $(\psi^\brak{p}_1)'(x)=-1/h_p$ and 
$(\psi^\brak{p}_2)'(x)=1/h_p$, we see that
\[
a^\brak{p}_{11}=a^\brak{p}_{22}
	=\int_{x_{p-1}}^{x_p}\frac{1}{h_p^2}\,dx
	=\frac{x_p-x_{p-1}}{h_p^2}=\frac{1}{h_p}
\]
and
\[
a^\brak{p}_{12}=a^\brak{p}_{21}
	=\int_{x_{p-1}}^{x_p}\frac{-1}{h_p^2}\,dx=\frac{-1}{h_p}.
\]
For the element mass matrix,
\[
c^\brak{p}_{11}=\int_{x_{p-1}}^{x_p}\biggl(\frac{x_p-x}{h_p}\biggr)^2\,dx
	=\biggl[-\frac{(x_p-x)^3}{3h_p^2}\biggr]_{x_{p-1}}^{x_p}
	=\frac{(x_p-x_{p-1})^3}{3h_p^2}=\frac{h_p}{3}
\]
and similarly
\[
c^\brak{p}_{22}=\int_{x_{p-1}}^{x_p}\biggl(\frac{x-x_{p-1}}{h_p}\biggr)^2\,dx
	=\biggl[-\frac{(x-x_{p-1})^3}{3h_p^2}\biggr]_{x_{p-1}}^{x_p}
	=\frac{(x_p-x_{p-1})^3}{3h_p^2}=\frac{h_p}{3},
\]
whereas, integrating by parts,
\begin{align*}
c^\brak{p}_{12}&=c^\brak{p}_{21}=\int_{x_{p-1}}^{x_p}
	\biggl(\frac{x_p-x}{h_p}\biggr)\biggl(\frac{x-x_{p-1}}{h_p}\biggr)\,dx\\
	&=\frac{1}{h_p^2}\biggl(
	\biggl[(x_p-x)\frac{(x-x_{p-1})^2}{2}\biggr]_{x_{p-1}}^{x_p}
	-\int_{x_{p-1}}^{x_p}(-1)\frac{(x-x_{p-1})^2}{2}\,dx\biggr)\\
	&=\frac{1}{2h_p^2}\int_{x_{p-1}}^{x_p}(x-x_{p-1})^2\,dx
	=\frac{1}{2h_p^2}\biggl[\frac{(x-x_{p-1})^3}{3}\biggr]_{x_{p-1}}^{x_p}
	=\frac{h_p}{6}.
\end{align*}
\end{example}

\begin{example}\label{example: assemble A}
Consider a uniform grid with $M=4$ subintervals, each of length~$h_m=1$,
and suppose $a(x)=1=c(x)$. Assembling the $5\times4$ stiffness 
matrix~$\boldsymbol{A}=[\boldsymbol{A}\free\quad\boldsymbol{A}\fix]$ amounts 
to computing
\[
\left[
\begin{array}{cccc|c}1&0&0&0&-1\\0&0&0&0&0\\0&0&0&0&0\\0&0&0&0&0\end{array}
\right]+\left[
\begin{array}{cccc|c}1&-1&0&0&0\\-1&1&0&0&0\\0&0&0&0&0\\0&0&0&0&0\end{array}
\right]+\left[
\begin{array}{cccc|c}0&0&0&0&0\\0&1&-1&0&0\\0&-1&1&0&0\\0&0&0&0&0\end{array}
\right]+\left[
\begin{array}{cccc|c}0&0&0&0&0\\0&0&0&0&0\\0&0&1&-1&0\\0&0&-1&1&0\end{array}
\right],
\]
so
\[
\boldsymbol{A}=\left[\begin{array}{cccc|c}
2&-1&0&0&-1\\ -1&2&-1&0&0\\ 0&-1&2&-1&0\\ 0&0&-1&1&0\end{array}\right],\qquad
\boldsymbol{A}\free=\begin{bmatrix}
2&-1&0&0\\ -1&2&-1&0\\ 0&-1&2&-1\\ 0&0&-1&1\end{bmatrix},\qquad
\boldsymbol{A}\fix=\begin{bmatrix}-1\\ 0\\ 0\\ 0\end{bmatrix}.
\]
Similarly, we find that the mass 
matrix~$\boldsymbol{C}=[\boldsymbol{C}\free\quad\boldsymbol{C}\fix]$ is given by
\[
\boldsymbol{C}=\frac{1}{6}\left[\begin{array}{cccc|c}
4&1&0&0&1\\ 1&4&1&0&0\\ 0&1&4&1&0\\ 0&0&1&2&0\end{array}\right],\qquad
\boldsymbol{C}\free=\frac{1}{6}\begin{bmatrix}
4&1&0&0\\ 1&4&1&0\\ 0&1&4&1\\ 0&0&1&2\end{bmatrix},\qquad
\boldsymbol{C}\fix=\frac{1}{6}\begin{bmatrix}1\\ 0\\ 0\\ 0\end{bmatrix}.
\]
\end{example}


\section{Quadratic elements}\label{sec: quadratic elements}

Let us again consider the self-adjoint, two-point boundary-value 
problem~\eqref{eq: self-adjoint mixed} but now let $V_h$ denote the vector 
space consisting of the continuous, piecewise-\emph{quadratic} functions with 
respect to the grid points~$x_p$.  Thus, a function~$v:[0,L]\to\mathbb{R}$ 
belongs to~$V_h$ iff there are polynomials $v_1$, $v_2$, \dots, $v_M$ of degree 
at most~$2$ such that \eqref{eq: v piecewise}~and \eqref{eq: v cts} hold.
We define the nodes
\[
\mathsf{n}^\brak{p}_1=x_{p-1},\qquad
\mathsf{n}^\brak{p}_2=\tfrac12(x_{p-1}+x_p),\qquad
\mathsf{n}^\brak{p}_3=x_p,
\]
and corresponding quadratic shape functions
\begin{align*}
\psi^\brak{p}_1(x)
	&=2h_p^{-2}(x-\mathsf{n}^\brak{p}_2)(x-\mathsf{n}^\brak{p}_3),\\
\psi^\brak{p}_2(x)
	&=4h_p^{-2}(x-\mathsf{n}^\brak{p}_1)(\mathsf{n}^\brak{p}_3-x),\\
\psi^\brak{p}_3(x)
	&=2h_p^{-2}(x-\mathsf{n}^\brak{p}_1)(x-\mathsf{n}^\brak{p}_2),
\end{align*}
satisfying
\[
\psi^\brak{p}_j(\mathsf{n}^\brak{p}_k)=\delta_{jk}.
\]
The global enumeration of the free nodes is
\[
\mathsf{n}_{2p-1}=\tfrac12(x_{p-1}+x_p)
\quad\text{and}\quad
\mathsf{n}_{2p}=x_p\quad\text{for $1\le p\le P$,}
\]
and the fixed node is $\mathsf{n}_{2P+1}=x_0$.  The connectivity 
matrix~$\boldsymbol{T}=[t_{jp}]$ is now $3\times P$, with $t_{jp}$ again given 
by~\eqref{eq: e pm def}.  For example, if $P=6$ then 
\[
\boldsymbol{T}=\begin{bmatrix}
13&2&4&6&8&10\\
 1&3&5&7&9&11\\
 2&4&6&8&10&12          
\end{bmatrix}.
\]

We define the \emph{reference element}~$[0,1]$ with \emph{reference nodes}
\[
\mathsf{n}_1=0,\qquad\mathsf{n}_2=\frac{1}{2},\qquad\mathsf{n}_3=1,
\]
and corresponding quadratic shape functions
\begin{equation}\label{eq: quadratic shape funcs 1d}
\Psi_1(\xi)=2(\xi-\tfrac12)(\xi-1),\qquad
\Psi_2(\xi)=4\xi(1-\xi),\qquad
\Psi_3(\xi)=2\xi(\xi-\tfrac12),
\end{equation}
satisfying $\Psi_j(\mathsf{n}_k)=\delta_{jk}$.  It is easy to verify that
\[
\Psi_j(\xi)=\psi^\brak{p}_j(x)\quad
	\text{if $x=x_{p-1}+\xi h_p$,}\quad\text{for $0\le\xi\le1$,}
\]
and since $dx/d\xi=h_p$, the chain rule implies that
\[
\Psi'_j(\xi)=h_p(\psi^\brak{p}_j)'(x).
\]
Thus, the entries of the $3\times3$ element stiffness matrix are given by
\begin{equation}\label{eq: a reference 1d}
a^\brak{p}_{jk}=\frac{1}{h_p}\int_0^1a(x_{p-1}+\xi h_p)
	\Psi'_k(\xi)\Psi'_j(\xi)\,d\xi,
\end{equation}
and those of the $3\times3$ element mass matrix by
\begin{equation}\label{eq: c reference 1d}
c^\brak{p}_{jk}=h_p\int_0^1c(x_{p-1}+\xi h_p)
	\Psi_k(\xi)\Psi_j(\xi)\,d\xi.
\end{equation}

\begin{example}
If $a(x)=1$~and $c(x)=1$, then
\begin{equation}\label{eq: element matrices quadratic 1d}
\boldsymbol{A}^\brak{p}=\frac{1}{3h_p}\begin{bmatrix}
 7&-8& 1\\
-8&16&-8\\
 1&-8& 7\end{bmatrix}
\quad\text{and}\quad
\boldsymbol{C}^\brak{p}=\frac{h_p}{30}\begin{bmatrix}
 4& 2&-1\\
 2&16& 2\\
-1& 2& 4 \end{bmatrix}
\end{equation}
\end{example}

To deal with non-constant coefficients $a(x)$~and $c(x)$, it is convenient to 
approximate the integrals \eqref{eq: a reference 1d}~and
\eqref{eq: c reference 1d} using a \emph{quadrature rule}
\[
\int_0^1 f(\xi)\,d\xi\approx\sum_{l=1}^J w_lf(\xi_l)
\]
with \emph{weights} $w_1$, $w_2$, \dots, $w_J$ and \emph{points}
\[
0\le\xi_1<\xi_2<\cdots<\xi_J\le1.
\]
Using such an approximation yields, for the element stiffness matrix,
\[
\boldsymbol{A}^\brak{p}\approx
	\widetilde{\boldsymbol{A}}^\brak{p}=[\tilde a_{jk}^\brak{p}]
\quad\text{where}\quad
\tilde a^\brak{p}_{jk}=\frac{1}{h_p}\sum_{j=1}^Jw_la(x_{p-1}+\xi_lh_p)
	\Psi'_k(\xi_l)\Psi'_j(\xi_l),
\]
and for the element mass matrix,
\[
\boldsymbol{C}^\brak{p}\approx
	\widetilde{\boldsymbol{C}}^\brak{p}=[\tilde c_{jk}^\brak{p}]
\quad\text{where}\quad
\tilde c^\brak{p}_{jk}=h_p\sum_{j=1}^Jw_lc(x_{p-1}+\xi_lh_p)
	\Psi_k(\xi_l)\Psi_j(\xi_l).
\]
With the help of connectivity matrix, the $3\times3$ element matrices 
$\boldsymbol{A}^\brak{p}$~and $\boldsymbol{C}^\brak{p}$ can be assembled into 
the $(2P)\times(2P+1)$ global matrices $\boldsymbol{A}$~and $\boldsymbol{C}$; 
see exercise~\ref{ex: assemble quadratic 1d}.

\section{Polynomial interpolation}

Let $\mathbb{P}_r$ denote the vector space of real polynomials of degree at 
most~$r$.  Let $x_1$, $x_2$, \dots, $x_{r+1}$ be distinct points in an 
interval~$[a,b]$, and let $f:[a,b]\to\mathbb{R}$ be a continuous function.  We 
say that a polynomial~$g\in\mathbb{P}_r$ \emph{interpolates} $f$ at the given 
points if
\begin{equation}\label{eq: g interpolates f}
g(x_j)=f(x_j)\quad\text{for $1\le j\le r+1$.}
\end{equation}
To see that such a $g$ exists, we define the \emph{Lagrange 
interpolation polynomials} $\ell_1$, $\ell_2$, \dots, $\ell_{r+1}$ by
\[
\ell_j(x)=\prod_{1\le k\le r+1, k\ne j}
	\frac{x-x_k}{x_j-x_k}\quad\text{for $1\le j\le r+1$,}
\]
and then define the \emph{linear interpolation operator}~$\mathcal{Q}_r$ by
\[
(\mathcal{Q}_rf)(x)=\sum_{j=1}^{r+1}f(x_j)\ell_j(x).
\]
Since $\ell_j\in\mathbb{P}_r$ and $\ell_j(x_k)=\delta_{jk}$, we see that
$\mathcal{Q}_rf\in\mathbb{P}_r$ interpolates~$f$, that is,
\[
(\mathcal{Q}_rf)(x_j)=f(x_j)\quad\text{for $1\le j\le r+1$.}
\]
Moreover, $\mathcal{Q}_rf$ is the only such interpolant, because if 
$g\in\mathbb{P}_r$ satisfies \eqref{eq: g interpolates f} then the difference
$g-\mathcal{Q}_rf$ is a polynomial satisfying
\[
(g-\mathcal{Q}_rf)(x_j)=g(x_j)-(\mathcal{Q}_rf)(x_j)=f(x_j)-f(x_j)=0
	\quad\text{for $1\le j\le r+1$,}
\]
so, for some constant~$C$,
\[
g(x)-(\mathcal{Q}_rf)(x)=C(x-x_1)(x-x_2)\cdots(x-x_{r+1}) 
	=Cx^{r+1}+\text{lower degree terms}.
\]
But $g-\mathcal{Q}_rf$ has degree at most~$r$, so $C=0$ and hence
$g(x)=(\mathcal{Q}_rf)(x)$ for all~$x$. In particular, since any element 
of~$\mathbb{P}_r$ interpolates itself, it follows that
\[
\mathcal{Q}_rf=f\quad\text{for every $f\in\mathbb{P}_r$,}
\]
and thus $\mathcal{Q}_r^2=\mathcal{Q}_r$.  Hence, $\mathcal{Q}_r$ is a 
\emph{projection} onto~$\mathbb{P}_r$.

Let 
\[
\pi_{r,y}(x)=\frac{(x-y)_+^r}{r!}=\begin{cases}
	(x-y)^r/r!&\text{if $a\le y\le x\le b$,}\\
	0&\text{if $a\le x<y\le b$.}
\end{cases}
\]
If $f$ is $C^{r+1}$ on~$[a,b]$ then, by Theorem~\ref{thm: Taylor remainder}, 
\[
f(x)=\sum_{k=0}^r\frac{f^{(k)}(a)}{k!}\,(x-a)^k
	+\int_a^b\pi_{r,y}(x)f^{(r+1)}(y)\,dy\quad\text{for $a\le x\le b$.}
\]
Since the sum on the right defines a function in~$\mathbb{P}_r$,
applying $\mathcal{Q}_r$ to both sides gives
\[
(\mathcal{Q}_rf)(x)=\sum_{k=0}^r\frac{f^{(k)}(a)}{k!}\,(x-a)^k
	+\int_a^b(\mathcal{Q}_r\pi_{r,y})(x)f^{(r+1)}(y)\,dy
\quad\text{for $a\le x\le b$.}
\]
Hence, the \emph{interpolation error} has the integral representation
\begin{equation}\label{eq: Qr f error}
(\mathcal{Q}_rf)(x)-f(x)=\int_a^b K_r(x,y)f^{(r+1)}(y)\,dy
\quad\text{for $a\le x\le b$,}
\end{equation}
where the \emph{Peano kernel} is defined by
\[
K_r(x,y)=(\mathcal{Q}_r\pi_{r,y})(x)-\pi_{r,y}(x)
	\quad\text{for $x$, $y\in[a,b]$.}
\]

\begin{figure}
\caption{Peano kernel for linear interpolation at $0$~and $1$
(Example~\ref{example: linear interp}.}
\label{fig: linear Peano}
\begin{center}
\includegraphics[scale=0.6]{../src/chap2/Linear_PeanoK_3d.pdf}
\includegraphics[scale=0.5]{../src/chap2/Linear_PeanoK_contour.pdf}
\end{center}
\end{figure}

\begin{example}\label{example: linear interp}
Consider \emph{linear interpolation}, that is, $r=1$, with interpolation points 
$x_1=a$~and $x_2=b$.  In this case, the Lagrange polynomials are
\[
\ell_1(x)=\frac{x-x_2}{x_1-x_2}=\frac{b-x}{b-a}
\quad\text{and}\quad
\ell_2(x)=\frac{x-x_1}{x_2-x_1}=\frac{x-a}{b-a},
\]
and
\[
(\mathcal{Q}_1f)(x)=f(a)\ell_1(x)+f(b)\ell_2(x)
	=f(a)\,\frac{b-x}{b-a}+f(b)\,\frac{x-a}{b-a}.
\]
In particular, since $\pi_{1,y}(a)=0$~and $\pi_{1,y}(b)=b-y$ for~$a\le y\le b$,
\[
(\mathcal{Q}_1\pi_{1,y})(x)=\pi_{1,y}(a)\ell_1(x)+\pi_{1,y}(b)\ell_2(x)
	=\frac{(b-y)(x-a)}{b-a}\quad\text{for $x$, $y\in[a,b]$,}
\]
so
\[
K_1(x,y)=\frac{(b-y)(x-a)}{b-a}-(x-y)\quad\text{for $a\le y\le x\le b$,}
\]
and
\[
K_1(x,y)=\frac{(b-y)(x-a)}{b-a}\quad\text{for $a\le x\le y\le b$.}
\]
Thus, $K_1(x,y)$ is piecewise linear in~$x$ for fixed~$y$, and vice versa, with
\[
K_1(a,y)=0=K_1(b,y),\qquad K_1(x,a)=0=K_1(x,b),\qquad
K_1(x,x)=\frac{(b-x)(x-a)}{b-a}.
\]
\end{example}

\begin{figure}
\caption{Peano kernel~\eqref{eq: quadratic Peano} for quadratic interpolation 
at $-1$, $0$~and $+1$.}\label{fig: quadratic Peano}
\begin{center}
\includegraphics[scale=0.6]{../src/chap2/Quadratic_PeanoK_3d.pdf}
\includegraphics[scale=0.5]{../src/chap2/Quadratic_PeanoK_contour.pdf}
\end{center}
\end{figure}

\begin{example}\label{example: quadratic interp}
Let $x_1=-1$, $x_2=0$~and $x_3=1$.  The quadratic Lagrange polynomials are
\[
\ell_1(x)=\frac{(x-0)(x-1)}{(-1-0)(-1-1)},\qquad
\ell_2(x)=\frac{(x+1)(x-1)}{(0+1)(0-1)},\qquad
\ell_3(x)=\frac{(x+1)(x-0)}{(1+1)(1-0)},
\]
that is,
\[
\ell_1(x)=\tfrac12x(x-1),\qquad
\ell_2(x)=(1+x)(1-x),\qquad
\ell_3(x)=\tfrac12x(x+1).
\]
Since $\pi_{2,y}(-1)=0$, $\pi_{2,y}(0)=\tfrac12(-y)_+^2$~and 
$\pi_{2,y}(1)=\tfrac12(1-y)^2$ for~$-1\le y\le 1$,
\begin{equation}\label{eq: quadratic Peano}
K_2(x,y)=\tfrac12(-y)_+^2\ell_2(x)+\tfrac12(1-y)^2\ell_3(x)-\tfrac12(x-y)_+^2.
\end{equation}
Observe that $\ell_1(-x)=\ell_3(x)$ and $\ell_2(-x)=\ell_2(x)$, so
\[
K_2(-x,-y)=\tfrac12(y)_+^2\ell_2(x)+\tfrac12(1+y)^2\ell_1(x)-\tfrac12(y-x)_+^2
\]
and thus
\[
K_2(x,y)+K_2(-x,-y)=\tfrac12y^2\ell_2(x)+\tfrac12(1-y)^2\ell_3(x)
	+\tfrac12(1+y)^2\ell_1(x)-\tfrac12(x-y)^2.
\]
We use the identity $\ell_1(x)+\ell_2(x)+\ell_3(x)=(\mathcal{Q}_21)(x)=1$ to
replace $\ell_2(x)$ with~$1-\ell_1(x)-\ell_3(x)$ and deduce that
\begin{align*}
K_2(x,y)+K_2(-x,-y)&=\tfrac12\bigl[y^2-(x-y)^2\bigr]
	+\tfrac12\bigl[(1+y)^2-y^2\bigr]\ell_1(x)
	+\tfrac12\bigl[(1-y)^2-y^2\bigr]\ell_3(x)\\
	&=\tfrac12x(2y-x)+\tfrac12(1+2y)\ell_1(x)+\tfrac12(1-2y)\ell_3(x)\\
	&=\tfrac12\bigl[\ell_1(x)+\ell_3(x)-x^2\bigr]
	+y\bigl[x+\ell_3(x)-\ell_1(x)\bigr]=0.
\end{align*}
Hence, $K_2$ possesses the symmetry property (Figure~\ref{fig: quadratic Peano})
\begin{equation}\label{eq: K2 symmetry}
K_2(-x,-y)=-K_2(x,y)\quad\text{for $x$, $y\in[-1,1]$.}
\end{equation}
\end{example}

Since the $j$th derivative~$\pi_{r,y}^{(j)}(x)=\pi_{r-j,y}(x)$, we see that
\[
\partial_x^jK_r(x,y)=(\mathcal{Q}_r\pi_{r,y})^{(j)}(x)-\pi_{r-j,y}(x)
\quad\text{for $0\le j\le r$,}
\]
and that the $j$th derivative of the interpolation error has the representation
\begin{equation}\label{eq: Qr f error deriv}
(\mathcal{Q}_rf-f)^{(j)}(x)=\int_a^b\partial_x^j K_r(x,y)f^{(r+1)}(y)\,dy
\quad\text{for $a\le x\le b$.}
\end{equation}


\section{Accuracy of piecewise-polynomial approximation}

Fix $r+1$~nodes in the reference element~$[0,1]$,
\begin{equation}\label{eq: ref nodes 1d}
0=\mathsf{n}_1<\mathsf{n}_2<\cdots<\mathsf{n}_{r+1}=1,
\end{equation}
and consider an element~$[x_{p-1},x_p]$ with length~$h_p=x_p-x_{p-1}$.  The 
change of variable,
\[
x = x_{p-1} + \xi h_p\quad\text{for $\xi\in[0,1]$,}
\]
defines an affine bijection~$\xi\mapsto x$ from the reference element~$[0,1]$ 
onto~$[x_{p-1},x_p]$.  We assume that the nodes in~$[x_{p-1},x_p]$ correspond 
to the reference nodes under this affine bijection, that is,
\[
\mathsf{n}_j^\brak{p}=x_{p-1}+\mathsf{n}_jh_p\quad\text{for $1\le j\le r+1$.}
\]
Given a function $f(x)$, we define $\hat f$ on the reference element by
\[
\hat f(\xi)=f(x)\quad\text{where $x=x_{p-1}+\xi h_p$,}
\]
and let $\mathcal{Q}_r^\brak{p}$~and $\widehat{\mathcal{Q}}_r$ denote 
polynomial interpolation operators for $[x_{p-1},x_p]$~and $[0,1]$, 
respectively, so that
\[
\mathcal{Q}_r^\brak{p}f\in\mathbb{P}_{r+1}\quad\text{and}\quad
(\mathcal{Q}_r^\brak{p}f)(\mathsf{n}_j^\brak{p})=f(\mathsf{n}_j^\brak{p})
\quad\text{for $1\le j\le r+1$,}
\]
and
\[
\widehat{\mathcal{Q}}_r\hat f\in\mathbb{P}_{r+1}\quad\text{and}\quad
\bigl(\widehat{\mathcal{Q}}_r\hat f\bigr)(\mathsf{n}_j)=\hat f(\mathsf{n}_j)
\quad\text{for $1\le j\le r+1$.}
\]
Notice that if we let $g(x)=(\mathcal{Q}_r^\brak{m}f)(x)$, then 
$\hat g(\xi)=\widehat{\mathcal{Q}}_r\hat f(\xi)$ because $\hat 
g\in\mathbb{P}_r$~and 
\[
\hat g(\mathsf{n}_j)=g(\mathsf{n}_j^\brak{m})=f(\mathsf{n}_j^\brak{m})
    =\hat f(\mathsf{n}_j)\quad\text{for $1\le j\le r+1$.}
\]
In other words, 
$\widehat{\mathcal{Q}_r^\brak{m}f}=\widehat{\mathcal{Q}}_r\hat f$.

\begin{theorem}
Assume that $f$ is $C^{r+1}$ on~$[x_{p-1},x_p]$, and let $K_r$ denote the Peano 
kernel for polynomial interpolation at the $r+1$~nodes of the reference 
element~$[0,1]$.  Then,
\[
\int_{x_{p-1}}^{x_p}|(f-\mathcal{Q}_r^\brak{p}f)^{(n)}(x)|^2\,dx
    \le C_{r,n}h_p^{2(r+1-n)}\int_{x_{p-1}}^{x_p}|\hat f^{(r+1)}(x)|^2\,dx
    \quad\text{for $0\le n\le r$,}
\]
where the constant
\[
C_{r,n}=\int_0^1\int_0^1|\partial_\xi^nK_r(\xi,\eta)|^2\,d\eta\,d\xi
\]
depends only on $n$~and the choice of the $r+1$ nodes~\eqref{eq: ref nodes 1d}.
\end{theorem}
\begin{proof}
Since $dx/d\xi=h_p$ we see that
\[
\hat f^{(n)}(\xi)=h_p^n f^{(n)}(x),
\]
so using \eqref{eq: Qr f error deriv} and the Cauchy--Schwarz inequality,
\begin{align*}
\bigl|(f-\mathcal{Q}_r^\brak{p}f)^{(n)}(x)\bigr|^2
    &=h_p^{-2n}\bigl|\bigl(\hat f-\widehat{Q}_r\hat f)(\xi)\bigr|^2
    =h_p^{-2n}\biggl|\int_0^1\partial_\xi^n K_r(\xi,\eta)
        \hat f^{(r+1)}(\eta)\, d\eta\biggr|^2\\
    &\le h_p^{-2n}\biggl(\int_0^1|\partial_\xi^nK_r(\xi,\eta)|^2\,d\eta\biggr)
    \biggl(\int_0^1|\hat f^{(r+1)}(\eta)|^2\,d\eta\biggr)
\end{align*}
for $0\le\xi\le1$.  Integrating with respect to~$x$, and noting that 
$dx=h_p\,d\xi$, we have
\begin{align*}
\int_{x_{p-1}}^{x_p}\bigl|(f-\mathcal{Q}_rf)^{(n)}(x)\bigr|^2\,dx
&=h_p^{1-2n}\int_0^1\bigl|\bigl(\hat f-\widehat{Q}_r\hat f)(\xi)\bigr|^2\,d\xi\\
&\le h_p^{1-2n}\biggl(
    \int_0^1\int_0^1|\partial_\xi^nK_r(\xi,\eta)|^2\,d\eta\,d\xi\biggr)
    \biggl(\int_0^1|\hat f^{(r+1)}(\eta)|^2\,d\eta\biggr),
\end{align*}
and 
\[
\int_0^1|\hat f^{(r+1)}(\eta)|^2\,d\eta=\int_{x_{p-1}}^{x_p}
    |h_n^{r+1}f^{(r+1)}(y)|^2h_p^{-1}\,dy,
\]
which implies the desired estimate.
\end{proof}

For any continuous~$v:[0,L]\to\mathbb{R}$ we define $\mathcal{Q}_{r,h}v$, the 
\emph{piecewise-polynomial interpolant} of degree at most~$r$, by
\[
(\mathcal{Q}_{r,h}v)(x)
    =(\mathcal{Q}_r^\brak{p}v)(x)\quad\text{for $x\in[x_{p-1},x_p]$;}
\]
since $\mathsf{n}^\brak{p-1}_{r+1}=x_{p-1}=\mathsf{n}^\brak{p}_1$ we see 
that
\[
(\mathcal{Q}_r^\brak{p-1}v)(x_{p-1})=v(x_{p-1})
    =(\mathcal{Q}_r^\brak{p}v)(x_{p-1})\quad\text{for~$2\le p\le P-1$,}
\]
and hence the function~$\mathcal{Q}_{r,h}v$ is continuous on~$[0,L]$.

\begin{theorem}
If $v$ is $C^{r+1}$ on~$[0,L]$, then
\[
\|v-\mathcal{Q}_{r,h}v\|_{L_2(0,L)}
    \le\sqrt{C_{r,0}}\,h^{r+1}\|v^{(r+1)}\|_{L_2(0,L)}
\]
and
\[
\|v'-(\mathcal{Q}_{r,h}v)'\|_{L_2(0,L)}
    \le\sqrt{C_{r,1}}\,h^r\|v^{(r+1)}\|_{L_2(0,L)}.
\]
\end{theorem}
\begin{proof}
We have
\begin{align*}
\|v-\mathcal{Q}_{r,h}v\|_{L_2(0,L)}^2
    &=\int_0^L|(v-\mathcal{Q}_{r,h}v)(x)|^2\,dx=\sum_{p=1}^P
    \int_{x_{p-1}}^{x_p}|(v-\mathcal{Q}_r^\brak{p}v)(x)|^2\,dx\\
    &\le\sum_{p=1}^P C_{r,0}h_p^{2(r+1)}\int_{x_{p-1}}^{x_p}
        |\hat v^{(r+1)}(x)|^2\,dx\\
    &\le C_{r,0}h^{2(r+1)}\sum_{p=1}^M\int_{x_{p-1}}^{x_p} 
        |\hat v^{(r+1)}(x)|^2\,dx
    =C_{r,0}h^{2(r+1)}\|v^{(r+1)}\|_{L_2(0,L)}^2,
\end{align*}
which proves the first estimate.  The second follows in the same way.
\end{proof}

\section{Sturm--Liouville problems}\label{sec: Sturm-Liouville}

We will now consider a Sturm--Liouville equation,
\begin{equation}\label{eq: Sturm Liouville ODE}
-\bigl(a(x)\phi'\bigr)'+\bigl(c(x)-\lambda b(x)\bigr)\phi=0\quad
	\text{for $0<x<L$,}
\end{equation}
subject to mixed boundary conditions
\begin{equation}\label{eq: Sturm Liouville bc}
\phi(0)=0\quad\text{and}\quad a(L)\phi'(L)=0.
\end{equation}
In addition to assuming that the leading coefficient~$a(x)$ 
satisfies~\eqref{eq: ellipticity 1d}, we require that
\begin{equation}\label{eq: b>0}
b(x)>0\quad\text{for $0<x<L$.}
\end{equation}
A non-trivial solution~$\phi=\phi(x)$ is said to an \emph{eigenfunction} and 
$\lambda$ is then the corresponding \emph{eigenvalue}.  

The ODE~\eqref{eq: Sturm Liouville ODE} can be written as
\[
\mathcal{L}\phi=\lambda b\phi,
\]
where the self-adjoint differential operator~$\mathcal{L}$ is again given 
by~\eqref{eq: L self-adjoint}.  Recalling the 
identity~\eqref{eq: Lu v by parts}, we see that any \emph{eigenpair} 
$(\phi,\lambda)$ satisfies, for any test function~$v$,
\[
\int_0^L\bigl(a(x)\phi'(x)v(x)+c(x)\phi(x)v(x)\bigr)\,dx
	=\lambda\int_0^Lb(x)\phi(x)v(x)\,dx
\quad\text{provided $v(0)=0$.}
\]

Let $V_h$ denote the continuous, piecewise-quadratic finite element space 
discussed in Section~\ref{sec: quadratic elements}, and put
\[
S_h=T_h=\{\,v\in V_h:v(0)=0\,\}.
\]
We seek an approximate eigenfunction~$\phi_h\in S_h$ and corresponding 
approximate eigenvalue~$\lambda_h$ such that
\[
\int_0^L\bigl(a(x)\phi_h'(x)v(x)+c(x)\phi_h(x)v(x)\bigr)\,dx
	=\lambda\int_0^Lb(x)\phi_h(x)v(x)\,dx
\quad\text{for all $v\in T_h$.}
\]
Let $\chi_1$, $\chi_2$, \dots, $\chi_{2P+1}$ be the nodal basis for~$V_h$, so 
that
\[
\chi_r(\mathsf{n}_s)=\delta_{rs}
	\quad\text{for $r$, $s\in\{1,2,\ldots, 2P+1\}$,}
\]
and thus, since $\phi_h(\mathsf{n}_{2P+1})=\phi_h(x_0)=\phi_h(0)=0$,
\[
\phi_h(x)=\sum_{s=1}^{2P}\Phi_s\chi_s(x)
	\quad\text{for $0\le x\le L$},
	\quad\text{where $\Phi_s=\phi_h(\mathsf{n}_s)$.}
\]
Choosing $v=\Phi_r$ we see that
\[
\sum_{k=1}^{2P}\bigl(a_{rs}+c_{rs}\bigr)\Phi_s
	=\lambda_h\sum_{s=1}^{2P}b_{rs}\Phi_s\quad\text{for $1\le r\le 2P$,}
\]
where
\[
a_{rs}=\int_0^La(x)\chi'_s(x)\chi'_r(x)\,dx,\qquad
c_{rs}=\int_0^Lc(x)\chi_s(x)\chi_r(x)\,dx,
\]
and
\[
b_{rs}=\int_0^Lb(x)\chi_s(x)\chi_r(x)\,dx.
\]
In this way, we obtain a $(2P)\times(2P)$ \emph{generalized algebraic 
eigenproblem},
\[
\bigl(\boldsymbol{A}+\boldsymbol{C}\bigr)\boldsymbol{\Phi}
	=\Lambda\boldsymbol{B}\boldsymbol{\Phi},
\]
where $\boldsymbol{A}=[a_{rs}]$, $\boldsymbol{C}=[c_{rs}]$, 
$\boldsymbol{B}=[b_{rs}]$, $\boldsymbol{\Phi}=[\Phi_s]$~and 
$\Lambda=\lambda_h$.  Each of the three matrices is real and symmetric, and our 
assumptions \eqref{eq: ellipticity 1d}~and \eqref{eq: b>0} on the coefficients 
$a(x)$~and $b(x)$ ensure that $\boldsymbol{A}$~and $\boldsymbol{B}$ are 
strictly positive-definite.  It follows that there exist eigenpairs 
$(\Phi_s,\Lambda_s)$ for~$1\le s\le 2P$, with
\[
\Lambda_1\le\Lambda_2\le\cdots\le\Lambda_{2P},\qquad
\boldsymbol{\Phi}_r^T\boldsymbol{A}\boldsymbol{\Phi}_s=\Lambda_s\delta_{rs},
\qquad
\boldsymbol{\Phi}_r^T\boldsymbol{B}\boldsymbol{\Phi}_s=\delta_{rs}.
\]
For a practical implementation, it is simpler to assemble $\boldsymbol{A}$, 
$\boldsymbol{B}$~and $\boldsymbol{C}$ from the corresponding element matrices
$\boldsymbol{A}^\brak{p}$, $\boldsymbol{B}^\brak{p}$~and 
$\boldsymbol{C}^\brak{p}$; see exercise \ref{ex: assemble quadratic 1d}~(iii).

\begin{Exercises}

\exercise
Consider the model problem~\eqref{eq: model 1d chap 2}.  Define a uniform 
mesh~$x_p=ph$ on~$[0,L]$ for $0\le p\le P=M$ where $h=\Delta x=L/P$.  Denote 
the finite difference solution by~$U^{\mathrm{d}}_p\approx u(x_p)$, and denote 
the nodal values of the piecewise-linear finite element solution 
by~$U^{\mathrm{e}}_p=u_h(x_p)\approx u(x_p)$.  Let 
\[
f^{\,\mathrm{d}}_p=f(x_p)\quad\text{and}\quad
f^{\,\mathrm{e}}_p=\frac{1}{h}\int_{x_{p-1}}^{x_p}f(x)\chi_p(x)\,dx
\quad\text{for $1\le p\le P-1$,}
\]
where $\chi_p$ is the $p$th piecewise-linear nodal basis function (that is, the 
``hat function'' centred on~$x_p$).  Let
\[
\boldsymbol{A}=\frac{1}{h^2}\begin{bmatrix}
 2&-1&      &      &      &\\
-1& 2&    -1&      &      &\\
  &  &\ddots&\ddots&\ddots&\\
  &  &      &    -1&     2&-1\\
  &  &      &      &    -1& 2
\end{bmatrix},\quad
\boldsymbol{f}^{\,\mathrm{d}}=\begin{bmatrix}
f^{\,\mathrm{d}}_1\\                              
f^{\,\mathrm{d}}_2\\                              
\vdots\\
f^{\,\mathrm{d}}_{P-2}\\                              
f^{\,\mathrm{d}}_{P-1}                              
\end{bmatrix},\quad
\boldsymbol{g}=\frac{1}{h^2}\begin{bmatrix}
\gamma_0\\ 0\\ \vdots\\ 0\\ \gamma_L                            
\end{bmatrix},
\]
so that $\boldsymbol{A}\boldsymbol{U}^{\mathrm{d}}=\boldsymbol{f}^{\,\mathrm{d}}
+\boldsymbol{g}$ as in~\eqref{eq: model 1d linear system}.
\begin{description}
\item{(i)} Show that $\boldsymbol{A}\boldsymbol{U}^{\mathrm{e}}
=\boldsymbol{f}^{\,\mathrm{e}}+\boldsymbol{g}$.
\item{(ii)} By applying Theorem~\ref{thm: discrete apriori 1D} to the finite 
difference equation satisfied by $U^{\mathrm{d}}_p-U^{\mathrm{e}}_p$, deduce 
that
\[
\max_{0\le p\le P}\bigl|U^{\mathrm{d}}_p-U^{\mathrm{e}}_p\bigr|
    \le\frac{L^2}{8a_{\mathrm{min}}}\,\max_{1\le q\le P-1}
    \bigl|f^{\,\mathrm{d}}_q-f^{\,\mathrm{e}}_q\bigr|.
\]
\item{(iii)}
Use Taylor expansion to show that
\[
f^{\,\mathrm{d}}_q-f^{\,\mathrm{e}}_q=\frac{1}{h^2}\int_{x_{q-1}}^{x_{q+1}}
    K_q(y)f''(y)\,dy
\]
where
\[
K_q(y)=\frac{1}{3!}\times\begin{cases}
        (y-x_{q-1})^3,&x_{q-1}<y<x_q,\\
        (x_{q+1}-y)^3,&x_q<y<x_{q+1}.
\end{cases}
\]
\item{(iv)}
Deduce using the Integral Mean Value Theorem that
\[
f^{\,\mathrm{d}}_q-f^{\,\mathrm{e}}_q=\frac{h^2}{4!}\,f''(\xi)
\quad\text{for some $\xi\in[x_{q-1},x_{q+1}]$,}
\]
and hence conclude that the finite difference and finite element solutions 
agree to~$O(h^2)$.
\end{description}

\exercise\label{ex: assemble quadratic 1d}
Recall the formulae \eqref{eq: a reference 1d}~and
\eqref{eq: c reference 1d} for the element stiffness and mass matrices using 
the quadratic shape functions~\eqref{eq: quadratic shape funcs 1d} on the 
reference element~$[0,1]$. 
\begin{description}
\item{(i)} Verify that $\Psi_3(1-\xi)=\Psi_1(\xi)$ and 
$\Psi_2(1-\xi)=\Psi_2(\xi)$.
\item{(ii)} Deduce that
$a^\brak{m}_{11}=a^\brak{m}_{33}$, $a^\brak{m}_{12}=a^\brak{m}_{32}$,
$c^\brak{m}_{11}=c^\brak{m}_{33}$, $c^\brak{m}_{12}=c^\brak{m}_{32}$.
\item{(iii)} Verify the formulae \eqref{eq: element matrices quadratic 1d}
for the element stiffness and mass matrices when $a(x)=1$~and $c(x)=1$.
\end{description}

\exercise
Consider the two-point boundary-value problem~\eqref{eq: self-adjoint mixed} 
where $\mathcal{L}$ is of the form~\eqref{eq: L self-adjoint}, and let
\[
L=2,\qquad a(x)=1,\qquad c(x)=0,\qquad f(x)=18.
\]
\begin{description}
\item{(i)} Suppose we take a uniform mesh with~$M=6$ subintervals and apply the 
finite element method using quadratic elements.  Write down the connectivity 
matrix~$[e_{pm}]$, the stiffness matrix~$\boldsymbol{A}$ and the load 
vector~$\boldsymbol{f}$. Hint: adapt the approach used in 
Example~\ref{example: assemble A} for piecewise-linear elements.
\item{(ii)} How do the boundary data $\gamma_0$~and $\gamma_L$ enter into the 
linear system for the nodal vector?
\item{(iii)} Formulate an algorithm to assemble the stiffness matrix for a 
general~$M$.  Hint: adapt Algorithm~\ref{alg: assemble A piecewise linear}.
\end{description}
\begin{ans}
(i) 
\begin{gather*}
\boldsymbol{T}=\begin{bmatrix}
13& 2& 4& 6& 8&10\\
 1& 3& 5& 7& 9&11\\
 2& 4& 6& 8&10&12
\end{bmatrix},\\
\boldsymbol{A}=\left[\begin{array}{cccccccccccc|c}
16&-8&  &  &  &  &  &  &  &  &  &  &-8\\
-8&14&-8& 1&  &  &  &  &  &  &  &  & 1\\
  &-8&16&-8&  &  &  &  &  &  &  &  &  \\
  & 1&-8&14&-8& 1&  &  &  &  &  &  &  \\
  &  &  &-8&16&-8&  &  &  &  &  &  &\\
  &  &  & 1&-8&14&-8& 1&  &  &  &  &  \\
  &  &  &  &  &-8&16&-8&  &  &  &  &  \\
  &  &  &  &  & 1&-8&14&-8& 1&  &  &  \\
  &  &  &  &  &  &  &-8&16&-8&  &  &  \\
  &  &  &  &  &  &  & 1&-8&14&-8& 1&  \\
  &  &  &  &  &  &  &  &  &-8&16&-8&  \\
  &  &  &  &  &  &  &  &  & 1&-8& 7&
\end{array}\right],\quad
\boldsymbol{f}=\left[\begin{array}{c}
4\\ 2\\ 4\\ 2\\ 4\\ 2\\ 4\\ 2\\ 4\\ 2\\ 4\\ 1
\end{array}\right].
\end{gather*}
(ii) $\boldsymbol{A}'\boldsymbol{U}'=\boldsymbol{f}-\gamma_0\boldsymbol{A}''
+\gamma_L\boldsymbol{e}_{2M}$\\
(iii)
\begin{algorithmic}
\State Allocate storage for 
$\boldsymbol{A}=[a_{jk}]\in\mathbb{R}^{(2M)\times(2M+1)}$ 
\For{$j=1:2M$}
    \For{$k=1:2M+1$}
        \State $a_{jk}=0$ 
    \EndFor
\EndFor
\For{$m=1:M$}
    \State Compute $\boldsymbol{A}^\brak{m}$ 
    \For{$p=1:3$}
        \State $j=e_{pm}$
        \If{$j\le2M$}
            \For{$q=1:3$}
                \State $k=t_{qm}$
                \State $a_{jk}\gets a_{jk}+a^\brak{m}_{pq}$
            \EndFor
        \EndIf
    \EndFor
\EndFor
\end{algorithmic}
\end{ans}

\exercise
Show that the Peano kernel for linear interpolation, discussed in 
Example~\ref{example: linear interp}, satisfies
\[
\int_a^b K_1(x,y)\,dy=\frac{1}{2}\,(x-a)(b-x)\quad\text{for $a\le x\le b$.}
\]
Hence use the integral mean value theorem to show that if~$f$ is $C^2$ 
on~$[a,b]$, then 
\[
(\mathcal{Q}_1f)(x)-f(x)=\frac{f''(\xi)}{2}\,(x-a)(b-x)
\]
for some~$\xi\in[a,b]$ (depending on~$x$).

\exercise
Let $x_1=-1$, $x_2=0$~and $x_3=+1$ as in 
Example~\ref{example: quadratic interp}.  
\begin{description}
\item{(i)} Show that if $g(x)=f(-x)$, then 
$(\mathcal{Q}_2g)(x)=(\mathcal{Q}_2f)(-x)$.
\item{(ii)} Deduce that if $f$ is $C^3$ on~$[-1,1]$, then
\[
\int_{-1}^1K_2(-x,y)f'''(y)\,dy=\int_{-1}^1K_2(x,y)g'''(y)\,dy
	\quad\text{for $-1\le x\le 1$.}
\]
\item{(iii)} Hence show that $K_2(-x,y)=-K_2(x,-y)$, in agreement 
with~\eqref{eq: K2 symmetry}.
\end{description}

\exercise
Sketch the graphs of the piecewise-quadratic nodal basis functions $\chi_1$, 
$\chi_2$, \dots, $\chi_{2P+1}$ used in section~\ref{sec: Sturm-Liouville}.

\end{Exercises}
