\chapter{General tridiagonal linear systems}

We consider solving an $n\times n$ linear system 
$\boldsymbol{A}\boldsymbol{x}=\boldsymbol{b}$ where the matrix~$\boldsymbol{A}$
is tridiagonal but not necessarily symmetric or positive-definite.  For concreteness,
suppose at first that $n=5$ and write
\[
\boldsymbol{A}=\begin{bmatrix}
\alpha_1&\gamma_1&        &        &\\
 \beta_1&\alpha_2&\gamma_2&        &\\
        & \beta_2&\alpha_3&\gamma_3&\\
        &        & \beta_3&\alpha_4&\gamma_4\\
        &        &        & \beta_4&\alpha_5
\end{bmatrix}
\]
We will reduce $\boldsymbol{A}$ to upper triangular form by Gaussian 
elimination.  

If $|\beta_1|\le|\alpha_1|$ then we define the \emph{Gauss transformation}
\[
\boldsymbol{M}_1=\begin{bmatrix}
     1& & & &\\
-\ell_1&1& & &\\
      & &1& &\\
      & & &1&\\
      & & & &1
\end{bmatrix}
\quad\text{where}\quad\ell_1=\frac{\beta_1}{\alpha_1},
\]
in order to eliminate the $21$-entry~$\beta_1$:
\[
\boldsymbol{M}_1\boldsymbol{A}=
\begin{bmatrix}
      1& & & &\\
-\ell_1&1& & &\\
       & &1& &\\
       & & &1&\\
       & & & &1
\end{bmatrix}
\begin{bmatrix}
\alpha_1&\gamma_1&        &        &\\
 \beta_1&\alpha_2&\gamma_2&        &\\
        & \beta_2&\alpha_3&\gamma_3&\\
        &        & \beta_3&\alpha_4&\gamma_4\\
        &        &        & \beta_4&\alpha_5
\end{bmatrix}
=\begin{bmatrix}
\alpha_1& \gamma_1&        &        &\\
        &\alpha_2'&\gamma_2&        &\\
        &  \beta_2&\alpha_3&\gamma_3&\\
        &         & \beta_3&\alpha_4&\gamma_4\\
        &         &        & \beta_4&\alpha_5
\end{bmatrix}
\]
where
\[
\alpha_2'=\alpha_2-\ell_1\gamma_1.
\]

However, if $|\alpha_1|<|\beta_1|$ then we swap the first two rows to move 
$\beta_1$ to the pivot position.  To perform this operation, we swap the first 
two rows of the $5\times5$ identity matrix to obtain a \emph{permutation matrix}
\[
\boldsymbol{P}_1=\begin{bmatrix}
 &1& & &\\
1& & & &\\
 & &1& &\\
 & & &1&\\
 & & & &1
\end{bmatrix},
\]
and observe that
\[
\boldsymbol{P}_1\boldsymbol{A}=
\begin{bmatrix}
 &1& & &\\
1& & & &\\
 & &1& &\\
 & & &1&\\
 & & & &1
\end{bmatrix}
\begin{bmatrix}
\alpha_1&\gamma_1&        &        &\\
 \beta_1&\alpha_2&\gamma_2&        &\\
        & \beta_2&\alpha_3&\gamma_3&\\
        &        & \beta_3&\alpha_4&\gamma_4\\
        &        &        & \beta_4&\alpha_5
\end{bmatrix}
=\begin{bmatrix}
 \beta_1&\alpha_2&\gamma_2&        &\\
\alpha_1&\gamma_1&        &        &\\
        & \beta_2&\alpha_3&\gamma_3&\\
        &        & \beta_3&\alpha_4&\gamma_4\\
        &        &        & \beta_4&\alpha_5
\end{bmatrix}.
\]
We let
\[
\ell_1=\frac{\alpha_1}{\beta_1},\quad
\alpha_1'=\beta,\quad\gamma_1'=\alpha_2,\quad\delta_1=\gamma_2,\quad
\alpha_2'=\gamma_1-\ell_1\alpha_2,\quad\gamma_2'=-\ell_1\gamma_2,
\]
so that
\[
\boldsymbol{M}_1\boldsymbol{P_1}\boldsymbol{A}
=\begin{bmatrix}
      1& & & &\\
-\ell_1&1& & &\\
       & &1& &\\
       & & &1&\\
       & & & &1
\end{bmatrix}
\begin{bmatrix}
 \beta_1&\alpha_2&\gamma_2&        &\\
\alpha_1&\gamma_1&        &        &\\
        & \beta_2&\alpha_3&\gamma_3&\\
        &        & \beta_3&\alpha_4&\gamma_4\\
        &        &        & \beta_4&\alpha_5
\end{bmatrix}
=\begin{bmatrix}
\alpha_1'&\gamma_1'& \delta_1&        &\\
         &\alpha_2'&\gamma_2'&        &\\
         & \beta_2 &\alpha_3 &\gamma_3&\\
         &         & \beta_3 &\alpha_4&\gamma_4\\
         &         &         & \beta_4&\alpha_5
\end{bmatrix}.
\]


