\chapter[Finite differences in 1D]{Finite differences for \\ 
stationary problems in 1D}
We begin our study of numerical methods for partial differential equations 
(PDEs) by treating the one-dimensional (1D) case, which means we deal only with 
an \emph{ordinary} differential equation (ODE).  Moreover, in this chapter we 
limit our attention to \emph{two-point boundary-value problems} for 
\emph{second-order} ODEs.  In fact, we will focus mostly on the very simple 
model problem on the interval~$[0,L]$,
\begin{equation}\label{eq: model 1d}
-u''=f(x)\quad\text{for $0<x<L$,}
	\quad\text{with $u(0)=\gamma_0$ and $u(L)=\gamma_L$,}
\end{equation}
in which the \emph{source term}~$f(x)$ and \emph{boundary data} $\gamma_0$, 
$\gamma_L$ are given, and we seek the \emph{unknown solution}~$u(x)$.

One can easily verify that for a constant source term~$f(x)=c$, the solution is 
given by
\[
u(x)=\frac{1}{L}\bigl((L-x)\gamma_0+x\gamma_L\bigr)+\frac{c}{2}\,x(L-x)
	\quad\text{for $0\le x\le L$.}
\]
For a general~$f$, the variation-of-parameters technique can be used to show 
that
\begin{equation}\label{eq: model 1d exact soln}
u(x)=\frac{L-x}{L}\biggl(\gamma_0+\int_0^x yf(y)\,dy\biggr)
	+\frac{x}{L}\biggl(\gamma_L+\int_x^L(L-y)f(y)\,dy\biggr)
	\quad\text{for $0\le x\le L$.}
\end{equation}

\begin{theorem}

\end{theorem}


\section{Second-order central difference}
To derive a finite difference approximation to the second derivative~$u''(x)$, 
we will use Taylor's theorem in the following form.

\begin{theorem}
Let $h>0$. If $f$ is $C^{k+1}$ on the closed interval~$[x,x+h]$ then
\[
f(x+h)=\sum_{j=0}^k\frac{1}{j!}\,f^{(j)}(x)\,h^j+(R_kf)(x,h)
\]
where the remainder term is given by
\[
(R_kf)(x,h)=\frac{1}{k!}\int_x^{x+h}(x+h-y)^kf^{(k+1)}(y)\,dy
\]
and satisfies
\[
|(R_kf)(x,h)|\le\frac{h^{k+1}}{(k+1)!}\,\max_{x\le y\le x+h}|f^{(k+1)}(y)|.
\]
Similarly, if $f$ is $C^{k+1}$ on~$[x-h,x]$, then
\[
f(x-h)=\sum_{j=0}^k\frac{(-1)^j}{j!}\,f^{(j)}(x)\,h^j+(R_kf)(x,-h)
\]
where 
\[
(R_kf)(x,-h)=\frac{(-1)^{k+1}}{k!}\int_{x-h}^x(y+h-x)^kf^{(k+1)}(y)\,dy
\]
and
\[
|(R_kf)(x,-h)|\le\frac{h^{k+1}}{(k+1)!}\,\max_{x-h\le y\le x}|f^{(k+1)}(y)|.
\]
\end{theorem}

In fact, a straight forward manipulation of such Taylor expansions shows that
\[
f''(x)=\frac{f(x+h)-2f(x)+f(x-h)}{h^2}+O(h^2)\quad\text{as $h\to0$;}
\]
more precisely, the following result holds.

\begin{theorem}
If $f$ is $C^4$ on the closed interval~$[x-h,x+h]$, then
\[
\biggl|f''(x)-\frac{f(x+h)-2f(x)+f(x-h)}{h^2}\biggr|\le\frac{h^2}{12}
	\max_{x-h\le y\le x+h}|f^{(4)}(y)|.
\]
\end{theorem}
\begin{proof}
Since
\[
f(x+h)=f(x)+f'(x)h+\tfrac12f''(x)h^2+\tfrac{1}{3!}f'''(x)h^3+(R_3f)(x,h)
\]
and
\[
f(x-h)=f(x)-f'(x)h+\tfrac12f''(x)h^2-\tfrac{1}{3!}f'''(x)h^3+(R_3f)(x,-h)
\]
we see that
\[
f(x+h)+f(x-h)=2f(x)+f''(x)h^2+(R_3f)(x,h)+(R_3f)(x,-h).
\]
Thus,
\[
\frac{f(x+h)-2f(x)+f(x-h)}{h^2}-f''(x)=\frac{(R_3f)(x,h)+(R_3f)(x,-h)}{h^2}
\]
and since
\begin{align*}
|(R_3f)(x,h)+(R_3f)(x,-h)|&\le
\frac{h^4}{4!}\max_{x\le y\le x+h}|f^{(4)}(y)|
+\frac{h^4}{4!}\max_{x-h\le y\le x}|f^{(4)}(y)|\\
	&\le\frac{2h^4}{4!}\,\max_{x-h\le y\le x+h}|f^{(4)}(y)|
\end{align*}
the result follows.
\end{proof}

To set up our finite difference scheme for~\eqref{eq: model 1d}, we choose a 
positive integer~$P$ and define a uniform grid on~$[0,L]$,
\[
x_p=p\,\Delta x\quad\text{for $0\le p\le P$,}
	\quad\text{where $\Delta x=\frac{L}{P}$.}
\]
In this way, we divide $[0,L]$ into $P$~subintervals, namely $[x_{p-1},x_p]$
for~$1\le p\le P$, each of length~$\Delta x$.  Our aim is to compute 
approximate values of the solution at each of the grid points,
\[
U_p\approx u(x_p)\quad\text{for $0\le p\le P$.}
\]
The approximation
\begin{equation}\label{eq: u'' approx}
\frac{u(x_{p+1})-2u(x_p)+u(x_{p-1})}{\Delta x^2}
	=\frac{u(x_p+\Delta x)-2u(x_p)+u(x_p-\Delta x)}{\Delta x^2}
	\approx u''(x_p)
\end{equation}
suggests the following discrete approximation to the ODE $-u''=f(x)$,
\begin{equation}\label{eq: model 1d discrete}
-\frac{U_{p+1}-2U_p+U_{p-1}}{\Delta x^2}=f(x_p)\quad\text{for $1\le p\le P-1$.}
\end{equation}
We can satisfy the boundary conditions $u(x_0)=u(0)=\gamma_0$ and 
$u(x_P)=u(L)=\gamma_L$ exactly, by putting
\[
U_0=\gamma_0\quad\text{and}\quad U_P=\gamma_L.
\]
When~$p=1$ in~\eqref{eq: model 1d discrete} we move $U_0=\gamma_0$ to the 
right-hand side, 
\[
\frac{2U_1-U_2}{\Delta x^2}=f(x_1)+\frac{\gamma_0}{\Delta x^2},
\]
and similarly when $p=P-1$ we move $U_P=\gamma_L$ to the right-hand side,
\[
\frac{-U_{P-2}+2U_{P-1}}{\Delta x^2}=f(x_{P-1})+\frac{\gamma_L}{\Delta x^2}.
\]
For example, if $P=6$ then
\begin{equation}\label{eq: model 1d linear system}
\frac{1}{\Delta x^2}\begin{bmatrix}
 2&-1&  & &\\
-1& 2&-1& &\\
  &-1& 2&-1&\\
  &  &-1& 2&-1\\
  &  &  &-1& 2
\end{bmatrix}
\begin{bmatrix}U_1\\ U_2\\ U_3\\ U_4\\ U_5\end{bmatrix}
=\begin{bmatrix}f_1\\ f_2\\ f_3\\ f_4\\ f_5 \end{bmatrix}
+\frac{1}{\Delta x^2}
\begin{bmatrix}\gamma_0\\ 0\\ 0 \\ 0\\ \gamma_L \end{bmatrix},
\end{equation}
where, for brevity, we have written $f_p=f(x_p)$.


\section{Symmetric tridiagonal linear systems}
How can we solve a symmetric, tridiagonal linear system such as the one
\eqref{eq: model 1d linear system} arising from the finite difference 
scheme~\eqref{eq: model 1d discrete}?  To discuss this problem, consider a 
$5\times5$ matrix of the form
\[
\boldsymbol{A}=\begin{bmatrix}
\alpha_1& \beta_1&        &        &\\
 \beta_1&\alpha_1& \beta_2&        &\\
        & \beta_2&\alpha_3&\beta_3 &\\
        &        & \beta_3&\alpha_4&\beta_4\\
        &        &        & \beta_4&\alpha_5
\end{bmatrix}.
\]
A standard algorithm involves computing $5\times5$~matrices 
$\boldsymbol{L}$~and $\boldsymbol{D}$ of the form
\[
\boldsymbol{L}=\begin{bmatrix}
     1&      &      &      &\\
\ell_1&     1&      &      &\\
      &\ell_2&     1&      &\\
      &      &\ell_3&     1&\\
      &      &      &\ell_4&1
  \end{bmatrix}
\quad\text{and}\quad
\boldsymbol{D}=\begin{bmatrix}
d_1&   &   &   &\\
   &d_2&   &   &\\
   &   &d_3&   &\\
   &   &   &d_4&\\
   &   &   &   &d_5
  \end{bmatrix}
\]
having the property that
\begin{equation}\label{eq: L D LT}
\boldsymbol{A}=\boldsymbol{L}\boldsymbol{D}\boldsymbol{L}^T.
\end{equation}
Given a right-hand side vector~$\boldsymbol{b}$, we can solve the linear system
$\boldsymbol{A}\boldsymbol{x}=\boldsymbol{b}$ by solving in sequence the three
linear systems
\begin{equation}\label{eq: LDLT systems}
\boldsymbol{L}\boldsymbol{z}=\boldsymbol{b},\qquad
\boldsymbol{D}\boldsymbol{y}=\boldsymbol{z},\qquad
\boldsymbol{L}^T\boldsymbol{x}=\boldsymbol{y},
\end{equation}
because it will then follow that
\[
\boldsymbol{A}\boldsymbol{x}
    =\boldsymbol{L}\boldsymbol{D}\boldsymbol{L}^T\boldsymbol{x}
    =\boldsymbol{L}\boldsymbol{D}\boldsymbol{y}
    =\boldsymbol{L}\boldsymbol{z}=\boldsymbol{b}.
\]
Since $\boldsymbol{L}$ is lower triangular and $\boldsymbol{D}$ is diagonal,
we can easily compute first $\boldsymbol{z}$, then $\boldsymbol{y}$ and finally 
$\boldsymbol{x}$.  To see how, we write out the equations in the $5\times5$ 
case:
\begin{align*}
      z_1    &=b_1,& d_1y_1&=z_1,& x_1+\ell_1 x_2&=y_1,\\
\ell_1z_1+z_2&=b_2,& d_2y_2&=z_2,& x_2+\ell_2 x_3&=y_2,\\
\ell_2z_2+z_3&=b_3,& d_3y_3&=z_3,& x_3+\ell_3 x_4&=y_3,\\
\ell_3z_3+z_4&=b_4,& d_4y_4&=z_4,& x_4+\ell_4 x_5&=y_4,\\
\ell_4z_4+z_5&=b_5,& d_5y_5&=z_5,& x_5           &=y_5.
\end{align*}
Thus, the steps of the computation are as follows:
\begin{align*}
z_1&=b_1,           & y_1&=z_1/d_1,& x_5&=y_5,\\
z_2&=b_2-\ell_1 z_1,& y_2&=z_2/d_2,& x_4&=y_4-\ell_4x_5,\\
z_3&=b_3-\ell_2 z_2,& y_3&=z_3/d_3,& x_3&=y_3-\ell_3x_4,\\
z_4&=b_4-\ell_3 z_3,& y_4&=z_4/d_4,& x_2&=y_2-\ell_2x_3,\\
z_5&=b_5-\ell_4 z_4,& y_5&=z_5/d_5,& x_1&=y_1-\ell_1x_2.\\
\end{align*}
Matrix multiplication gives
\[
\boldsymbol{L}\boldsymbol{D}\boldsymbol{L}^T=\begin{bmatrix}
      d_1&      \ell_1d_1&               &               &         \\
\ell_1d_1&d_2+\ell_1^2d_1&      \ell_2d_2&               &         \\
         &      \ell_2d_2&d_3+\ell_2^2d_2&      \ell_3d_3&         \\
         &               &      \ell_3d_3&d_4+\ell_3^2d_3&\ell_4d_4\\
         &               &               &      \ell_4d_4&d_5+\ell_4^2d_4
\end{bmatrix},
\]
so the factorization~\eqref{eq: L D LT} requires that
\begin{align*}
     \alpha_1&=d_1,      &        &                 &&&&&&\\
      \beta_1&=\ell_1d_1,&\alpha_2&=d_2+\ell_1^2d_1,&&&&&&\\
    &&\beta_2&=\ell_2d_2,&\alpha_3&=d_3+\ell_2^2d_2,&&&&\\
  &&&&\beta_3&=\ell_3d_3,&\alpha_4&=d_4+\ell_3^2d_3,&&\\
&&&&&&\beta_4&=\ell_4d_4,&\alpha_5&=d_5+\ell_4^2d_4.
\end{align*}
Thus, the steps of the computation are as follows:
\begin{align*}
         d_1&=\alpha_1,   &   &                      &&&&&&\\
      \ell_1&=\beta_1/d_1,&d_2&=\alpha_2-\ell_1^2d_1,&&&&&&\\
    &&\ell_2&=\beta_2/d_2,&d_3&=\alpha_3-\ell_2^2d_2,&&&&\\
  &&&&\ell_3&=\beta_3/d_3,&d_4&=\alpha_4-\ell_3^2d_3,&&\\
&&&&&&\ell_4&=\beta_4/d_4,&d_5&=\alpha_5-\ell_4^2d_4.
\end{align*}
In the general $n\times n$~case,
Algorithm~\ref{alg: LDLT} computes the factorization~\eqref{eq: L D LT}
and Algorithm~\ref{alg: solve symmetric tridiagonal} 
uses this factorization to solve the linear 
system~$\boldsymbol{A}\boldsymbol{x}=\boldsymbol{b}$. 

The derivations above show that if the factorization \eqref{alg: LDLT} exists,
then it is unique, that is, $\boldsymbol{L}$ and $\boldsymbol{D}$ are uniquely 
determined by~$\boldsymbol{A}$.  Unfortunately,
Algorithm~\ref{alg: LDLT} can break down because one of the $d_j$ is zero, as 
the following example shows.

\begin{example}
The $3\times3$ matrix
\[
\boldsymbol{A}=\begin{bmatrix}1&1&0\\1&1&2\\ 0&2&0 \end{bmatrix}
\]
is non-singular with inverse
\[
\boldsymbol{A}^{-1}=\begin{bmatrix}1&0&-1/2\\ 0&0&1/2\\ -1/2&1/2&0\end{bmatrix},
\]
but Algorithm~\ref{alg: LDLT} produces
\[
d_1=1,\quad \ell_1=1,\quad d_2=0,
\]
and then breaks down because $\ell_2=2/0$.
\end{example}

However, the following result can be shown.

\begin{theorem}
If the $n\times n$, symmetric, tridiagonal matrix $\boldsymbol{A}$ is 
positive-definite, then the factorization~\eqref{eq: L D LT} exists and $d_j>0$ 
for $1\le j\le n$.
\end{theorem}

\begin{algorithm}
\caption{Compute the factorization \eqref{eq: L D LT} for a symmetric, 
tridiagonal matrix $\boldsymbol{A}$.}
\label{alg: LDLT}
\begin{algorithmic}
\Require{$\boldsymbol{\alpha}=[\alpha_1,\alpha_2,\ldots,\alpha_n]$ is the main 
diagonal of $\boldsymbol{A}$.}
\Require{$\boldsymbol{\beta}=[\beta_1,\beta_2,\ldots,\beta_{n-1}]$ is the 
off-diagonal of $\boldsymbol{A}$.}
\Statex
\Function{Factorize}{$\boldsymbol{\alpha}, \boldsymbol{\beta}$}
\State Allocate storage for $\boldsymbol{d}=[d_1,d_2,\ldots, d_n]$ and
$\boldsymbol{\ell}=[\ell_1,\ell_2,\ldots,\ell_{n-1}]$
\State $d_1=\alpha_1$
\For{$j=1:n-1$}
\State $\ell_j=\beta_j/d_j$
\State $d_{j+1}=\alpha_{j+1}-\ell_j^2d_j$
\EndFor
\State\Return{$\boldsymbol{d}$, $\boldsymbol{\ell}$}
\EndFunction
\end{algorithmic}
\end{algorithm}

\begin{algorithm}
\caption{Solve a symmetric, tridiagonal linear system 
$\boldsymbol{A}\boldsymbol{x}=\boldsymbol{b}$ given the 
factorization~\eqref{eq: L D LT}.}
\label{alg: solve symmetric tridiagonal}
\begin{algorithmic}
\Require{$\boldsymbol{b}=[b_1,b_2,\ldots,b_n]$ the right-hand side vector.}
\Require{$\boldsymbol{d}=[d_1, d_2, \ldots, d_n]$ the main diagonal of 
$\boldsymbol{D}$.}
\Require{$\boldsymbol{\ell}=[\ell_1,\ell_2,\ldots,\ell_{n-1}]$ the first 
sub-diagonal of~$\boldsymbol{L}$.}
\Statex
\Function{Solve}{$\boldsymbol{b},\boldsymbol{d},\boldsymbol{\ell}$}
\State Allocate storage for $\boldsymbol{x}=[x_1,x_2,\ldots,x_n]$,
$\boldsymbol{y}=[y_1,y_2,\ldots,y_n]$ and $\boldsymbol{z}=[z_1,z_2,\ldots,z_n]$.
\State $z_1=b_1$
\For{$j=1:n-1$}
\State $z_{j+1}=b_{j+1}-\ell_jz_j$
\EndFor
\For{$j=1:n$}
\State $y_j=z_j/d_j$
\EndFor
\State $x_n=y_n$
\For{$j=n-1:-1:1$}
\State $x_j=y_j-\ell_jx_{j+1}$
\EndFor
\State\Return{$\boldsymbol{x}$}
\EndFunction
\end{algorithmic}
\end{algorithm}

\section{General two-point boundary-value problem}

Now consider a general second-order linear differential operator,
\[
\mathcal{L}u=-a(x)u''+b(x)u'+c(x)u,
\]
where, for simplicity, we will assume that the coefficients $a$, $b$ and 
$c$ are continuous on~$[0,L]$, and that the \emph{leading coefficient}~$a$ is 
\emph{strictly positive} on~$[0,L]$, say
\[
a(x)\ge a_{\min}>0\quad\text{for $0\le x\le L$.}
\]
The general \emph{two-point boundary-value problem} is to find $u=u(x)$ 
satisfying
\begin{equation}\label{eq: two-point bvp}
\begin{aligned}
\mathcal{L}u&=f(x)&&\text{for $0<x<L$,}\\
\alpha_0u'+\beta_0u&=\gamma_0&&\text{at $x=0$,}\\
\alpha_Lu'+\beta_Lu&=\gamma_L&&\text{at $x=L$,}
\end{aligned}
\end{equation}
where, for the boundary conditions to make sense, we assume that at least one 
of $\alpha_0$~and $\beta_0$ is not zero, and likewise at least one of  
$\alpha_L$~and $\beta_L$ is not zero.  For simplicity, we will also assume that 
the function~$f$ is continuous on~$[0,L]$.

To construct a finite difference scheme for~\eqref{eq: two-point bvp}, we need 
to approximate the first derivative~$u'$.  Once again, we apply Taylor 
expansions, this time showing that
\[
\frac{f(x+h)-f(x-h)}{2h}=f'(x)+O(h^2)\quad\text{as $h\to0$.}
\]
More precisely, the following holds.

\begin{theorem}
If $f$ is $C^3$ on the closed interval~$[x-h,x+h]$, then
\[
\biggl|f'(x)-\frac{f(x+h)-f(x-h)}{2h}\biggr|
	\le\frac{h^2}{6}\,\max_{x-h\le y\le x+h}|f'''(y)|.
\]
\end{theorem}
\begin{proof}
Since
\[
f(x+h)=f(x)+f'(x)h+\tfrac12f''(x)h^2+(R_2f)(x,h)
\]
and
\[
f(x-h)=f(x)-f'(x)h+\tfrac12f''(x)h^2+(R_2f)(x,-h),
\]
we have
\[
f(x+h)-f(x-h)=2f'(x)h+(R_2f)(x,h)-(R_2f)(x,-h).
\]
Thus,
\[
\frac{f(x+h)-f(x-h)}{2h}-f'(x)=\frac{(R_2f)(x,h)-(R_2f)(x,-h)}{2h},
\]
and since
\begin{align*}
\bigl|(R_2f)(x,h)-(R_2f)(x,-h)\bigr|
	&\le\frac{h^3}{3!}\max_{x\le y\le x+h}|f'''(y)|
 	   +\frac{h^3}{3!}\max_{x-h\le y\le x}|f'''(y)|\\
	&\le\frac{2h^3}{3!}\max_{x-h\le y\le x+h}|f'''(y)|,
\end{align*}
the result follows at once.
\end{proof}

The approximations \eqref{eq: u'' approx}~and
\[
\frac{u(x_{p+1})-u(x_{p-1})}{2\,\Delta x}
	=\frac{u(x_p+\Delta x)-u(x_p-\Delta x)}{2\,\Delta x}
	\approx u'(x_p)
\]
suggest the following discrete approximation to the ODE~$\mathcal{L}u=f(x)$,
\begin{equation}\label{eq: Lu=f finite diff}
-a_p\,\frac{U_{p+1}-2U_p+U_{p-1}}{\Delta x^2}
	+b_p\,\frac{U_{p+1}-U_{p-1}}{2\,\Delta x}+c_pU_p=f_p,
\end{equation}
where we have used the abbreviations $a_p=a(x_p)$, $b_p=b(x_p)$, $c_p=c(x_p)$
and $f_p=f(x_p)$.  The boundary conditions in~\eqref{eq: two-point bvp} can be 
approximated by
\[
\alpha_0\,\frac{U_1-U_{-1}}{2\,\Delta x}+\beta_0U_0=\gamma_0
\quad\text{and}\quad
\alpha_L\,\frac{U_{P+1}-U_{P-1}}{2\,\Delta x}+\beta_LU_P=\gamma_L.
\]
There are four cases to consider.
\begin{enumerate}
\item If $\alpha_0=0$ and $\alpha_L=0$, then the unknowns are
$U_1$, $U_2$, \dots, $U_{P-1}$, with
\[
U_0=\gamma_0/\beta_0\quad\text{and}\quad U_P=\gamma_L/\beta_L.
\]
\item If $\alpha_0\ne0$ and $\alpha_L=0$, then the unknowns are
$U_0$, $U_1$, \dots, $U_{P-1}$, with 
\[
U_{-1}=U_1+2\,\Delta x\,(\beta_0U_0-\gamma_0)\quad\text{and}\quad
U_P=\gamma_L/\beta_L.
\]
\item If $\alpha_0=0$ and $\alpha_L\ne0$, then the unknowns are
$U_1$, $U_2$, \dots, $U_P$, with 
\[
U_0=\gamma_0/\beta_0\quad\text{and}\quad
U_{P+1}=U_{P-1}+2\,\Delta x(\gamma_L-\beta_LU_P)/\alpha_L.
\]
\item If $\alpha_0\ne0$ and $\alpha_L\ne0$, then the unknowns are
$U_0$, $U_1$, \dots, $U_P$, with
\[ 
U_{-1}=U_1+2\,\Delta x\,(\beta_0U_0-\gamma_0)\quad\text{and}\quad
U_{P+1}=U_{P-1}+2\,\Delta x(\gamma_L-\beta_LU_P)/\alpha_L.
\]
\end{enumerate}

In case~1, we use \eqref{eq: Lu=f finite diff} for $1\le p\le P-1$.
The first equation is
\[
-a_1\,\frac{U_2-2U_1+U_0}{\Delta x^2}+b_1\,\frac{U_2-U_0}{2\,\Delta x}
	+c_1U_1=f_1
\]
so $U_0=\gamma_0/\beta_0$ gives
\[
a_1\,\frac{2U_1-U_2}{\Delta x^2}+b_1\,\frac{U_2}{2\,\Delta x}+c_1U_1
	=f_1+\biggl(\frac{a_1}{\Delta x^2}+\frac{b_1}{2\,\Delta x}\biggr)
	\frac{\gamma_0}{\beta_0}.
\]
Similarly, the last equation is
\[
-a_{P-1}\,\frac{U_P-2U_{P-1}+U_{P-2}}{\Delta x^2}
	+b_{P-1}\,\frac{U_P-U_{P-2}}{2\,\Delta x}+c_{P-1}U_{P-1}=f_{P-1},
\]
so $U_P=\gamma_L/\beta_L$ gives
\[
a_{P-1}\,\frac{-U_{P-2}+2U_{P-1}}{\Delta x^2}
	-b_{P-1}\,\frac{U_{P-2}}{2\,\Delta x}+c_{P-1}U_{P-1}
	=f_{P-1}+\biggl(\frac{a_{P-1}}{\Delta x^2}-\frac{b_{P-1}}{2\,\Delta x}
	\biggr)\,\frac{\gamma_L}{\beta_L}.
\]
For example, if $P=6$ we obtain a $5\times5$ linear system
\[
\boldsymbol{A}\boldsymbol{U}+\boldsymbol{B}\boldsymbol{U}
	+\boldsymbol{C}\boldsymbol{U}=\boldsymbol{f}+\boldsymbol{g},
\]
where
\begin{gather*}
\boldsymbol{A}=\frac{1}{\Delta x^2}\begin{bmatrix}
2a_1&-a_1&    &    &    \\
-a_2&2a_2&-a_2&    &    \\
    &-a_3&2a_3&-a_3&    \\
    &    &-a_4&2a_4&-a_4\\
    &    &    &-a_5&2a_5\\
               \end{bmatrix},\qquad
\boldsymbol{B}=\frac{1}{2\,\Delta x}\begin{bmatrix}
   0& b_1&    &    &    \\
-b_2&   0& b_2&    &    \\
    &-b_3&   0& b_3&    \\
    &    &-b_4&   0& b_4\\
    &    &    &-b_5&   0\end{bmatrix},\\
\boldsymbol{C}=\begin{bmatrix}
c_1&   &   &   &   \\
   &c_2&   &   &   \\
   &   &c_3&   &   \\
   &   &   &c_4&   \\
   &   &   &   &c_5
\end{bmatrix},\quad
\boldsymbol{U}=\begin{bmatrix} U_1\\ U_2\\ U_3\\ U_4\\ U_5\end{bmatrix},\quad
\boldsymbol{f}=\begin{bmatrix} f_1\\ f_2\\ f_3\\ f_4\\ f_5\end{bmatrix},\quad
\boldsymbol{g}=\begin{bmatrix} g_1\\ 0\\ 0\\ 0\\ g_5 \end{bmatrix},
\end{gather*}
where
\[
g_1=\biggl(\frac{a_1}{\Delta x^2}+\frac{b_1}{2\,\Delta x}\bigr)
	\frac{\gamma_0}{\beta_0}
\quad\text{and}\quad
g_5=\biggl(\frac{a_{P-1}}{\Delta x^2}-\frac{b_{P-1}}{2\,\Delta x}
	\biggr)\,\frac{\gamma_L}{\beta_L}.
\]








\section{Maximum principle}

\begin{lemma}\label{lem: Lu<0}
Assume that $u$ is $C^2$ on the open interval~$(0,L)$.
If $c\ge0$ and $\mathcal{L}u<0$ on~$(0,L)$, then $u$ cannot attain a 
non-negative local maximum in~$(0,L)$.
\end{lemma}
\begin{proof}
Suppose for a contradiction that there exist $x_0\in(0,L)$~and $\delta>0$ such
that
\[
u(x_0)\ge0\quad\text{and}\quad
\text{$u(x)\le u(x_0)$ for~$x\in(x_0-\delta,x_0+\delta)\subseteq(0,L)$.}
\]
Since $u$ has an interior local maximum at~$x_0$, it follows that 
$u'(x_0)=0$~and $u''(x_0)\le0$ so 
\[
(\mathcal{L}u)(x_0)=-a(x_0)u''(x_0)+c(x_0)u(x_0)\ge 0,
\]
contradicting our assumption that $\mathcal{L}u<0$ on~$(0,L)$.
\end{proof}

\begin{theorem}
Assume that $u$ is continuous on the closed interval~$[0,L]$ and is $C^2$ on 
the open interval~$(0,L)$. If $c\ge0$ and $\mathcal{L}u\le0$ on~$(0,L)$, then
\[
u(x)\le\max\{u^+(0),u^+(L)\}\quad\text{for $0<x<L$.}
\]
\end{theorem}
\begin{proof}
Let $\epsilon>0$ and put $w(x)=u(x)+\epsilon e^{\mu x}$.  Since
\[
(\mathcal{L}w)(x)
	=(\mathcal{L}u)(x)+\epsilon\bigl[-a(x)\mu^2+b(x)\mu+c(x)\bigr]e^{\mu x}
	\le-\epsilon\bigl[a(x)\mu^2-\mu\|b\|_\infty-\|c\|_\infty\bigr]e^{\mu x}
\]
by choosing $|\mu|$ sufficiently large we can ensure that $\mathcal{L}w<0$
on~$(0,L)$.  By Lemma~\ref{lem: Lu<0}, the function~$w$ cannot attain a 
non-negative local maximum in~$(0,L)$, implying that 
\[
u(x)\le w(x)\le w^+(x)\le\max\{w^+(0),w^+(L)\}
	\le\max\{u^+(0)+\epsilon,u^+(L)+\epsilon e^{\mu L}\}
\]
for~$0<x<L$.  Since this inequality holds for any~$\epsilon>0$, the result 
follows.
\end{proof}

\begin{theorem}
Assume that $u$ is continuous on the closed interval~$[0,L]$ and is $C^2$ on 
the open interval~$(0,L)$. If $c\ge0$ and $\mathcal{L}u=f$ on~$(0,L)$, then
\[
\max_{[0,L]}u\le\max\{u^+(0),u^+(L)\}+\cosh(\mu L/2)\max_{[0,L]}f^+
\]
and
\[
\min_{[0,L]}u\ge\min\{u^-(0),u^-(L)\}+\cosh(\mu L/2)\max_{[0,L]}f^-,
\]
where $\mu=\max\bigl\{1,(1+\|b\|_\infty)/a_{\min}\bigr\}$.
\end{theorem}
\begin{proof}
Let
\[
w(x)=\max\{u^+(0),u^+(L)\}+v(x)\,\max_{[0,L]}f
\quad\text{where}\quad
v(x)=\cosh(\mu L/2)-\cosh\mu(x-L/2),
\]
and observe that $v\ge0$ on~$[0,L]$ with
\begin{align*}
(\mathcal{L}v)(x)&=a(x)\mu^2\cosh\mu(x-L/2)
	-b(x)\mu\sinh\mu(x-L/2)+c(x)v(x)\\
	&\ge\bigl(a_{\min}\mu^2-\|b\|_\infty\mu\bigr)\cosh\mu(x-L/2)
	\ge(a_{\min}\mu-\|b\|_\infty)\mu\ge1
\end{align*}
for $0<x<L$, so
\[
\mathcal{L}(u-w)=f-c(x)\max\{u^+(0),u^+(L)\}-(\mathcal{L}v)\max_{[0,L]}f^+
	\le0\quad\text{on $(0,L)$.}
\]
By Lemma~\ref{lem: Lu<0},
\[
u(x)-w(x)\le\max\{(u-w)^+(0),(u-w)^+(L)\}\quad\text{for $0<x<L$,}
\]
and since $v(0)=0=v(L)$ we see that 
\[
(u-w)(0)=u(0)-\max\{u^+(0),u^+(L)\}\le0
\]
and
\[
(u-w)(L)=u(L)-\max\{u^+(0),u^+(L)\}\le0.
\]
Thus, $u-w\le0$ on~$(0,L)$, and therefore 
$\max_{[0,L]}u\le\max_{[0,L]}w(x)=w(L/2)$, proving the first inequality.  
The second follows because $\mathcal{L}(-u)=-f$, $(-u)^+=-u^-$~and
$(-f)^+=-f^-$.
\end{proof}
\begin{corollary}\label{cor: Lu=f apriori infty}
$\|u\|_\infty\le\max\{|u(0)|,|u(L)|\}+\cosh(\mu L/2)\|f\|_\infty$.
\end{corollary}

Now consider the two-point boundary-value problem~\eqref{eq: two-point bvp} in 
the case when $\alpha_0=0=\alpha_L$ and $\beta_0=1=\beta_L$:
\begin{equation}\label{eq: Lu=f Dirichlet}
\mathcal{L}u=f\quad\text{on $(0,L)$,}
	\quad\text{with $u(0)=\gamma_0$ and $u(L)=\gamma_L$.}
\end{equation}
The next theorem shows that this problem is \emph{well-posed}, that is,
\begin{enumerate}
\item a unique solution exists for each choice of the data $f$, $\gamma_0$~and 
$\gamma_L$, and 
\item small changes in the data lead to only small changes in the solution.
\end{enumerate}
To explain the second part, suppose that we perturb the data to $\tilde f$, 
$\tilde\gamma_0$ and $\tilde\gamma_L$, and let $\tilde u$ denote the solution 
of the resulting perturbed problem,
\[
\mathcal{L}\tilde u=\tilde f\quad\text{on $(0,L)$,}
	\quad\text{with $\tilde u(0)=\tilde\gamma_0$ 
and $\tilde u(L)=\tilde\gamma_L$.}
\]
If w denote the changes in the solution and the data by
\[
\delta u=\tilde u-u,\quad\delta f=\tilde f-f,\quad
\delta\gamma_0=\tilde\gamma_0-\gamma_0,\quad
\delta\gamma_L=\tilde\gamma_L-\gamma_L,
\]
then we need to show that $\delta u$ is small whenever $\delta f$, 
$\delta\gamma_0$~and $\delta\gamma_L$ are all small.

\begin{theorem}
If $c\ge0$ on~$(0,L)$, then \eqref{eq: Lu=f Dirichlet} has a unique 
solution~$u$.  This solution is continuous on~$[0,L]$ and is $C^2$ on~$(0,L)$.
Moreover, when the problem is perturbed as described above, 
\[
\|\delta u\|_\infty\le\max\{|\delta\gamma_0|,|\delta\gamma_L|\}
	+\cosh(\mu L/2)\|\delta f\|_\infty.
\]
\end{theorem}
\begin{proof}
We will not prove the hard part, namely existence.  Since the $\mathcal{L}$ is 
linear, the proof of uniqueness follows easily from 
Corollary~\ref{cor: Lu=f apriori infty}.  In fact, suppose that $u_1$~and $u_2$ 
are solutions, that is,
\[
\mathcal{L}u_1=f=\mathcal{L}u_2\quad\text{on $(0,L)$,}
	\quad\text{with $u_1(0)=\gamma_0=u_2(0)$ and $u_1(L)=\gamma_L=u_2(L)$.}
\]
The difference $v=u_1-u_2$ satisfies
\[
\mathcal{L}v=\mathcal{L}(u_1-u_2)=\mathcal{L}u_1-\mathcal{L}u_2=f-f=0
	\quad\text{on $(0,L)$},
\]
with $v(0)=u_1(0)-u_2(0)=\gamma_0-\gamma_0=0$ and
$v(L)=u_1(L)-u_2(L)=\gamma_L-\gamma_L=0$, so $\|v\|_\infty\le0$ by 
Corollary~\ref{cor: Lu=f apriori infty}, which means that $u_1=u_2$ on~$[0,L]$.

Similarly, since $\mathcal{L}$ is linear,
\[
\mathcal{L}\delta u=\delta f\quad\text{on $(0,L)$,}
	\quad\text{with $\delta u(0)=\delta\gamma_0$ 
and $\delta u(L)=\delta\gamma_L$,}
\]
and Corollary~\ref{cor: Lu=f apriori infty} implies the desired estimate 
for~$\|\delta u\|_\infty$.
\end{proof}

\section{An error bound}
