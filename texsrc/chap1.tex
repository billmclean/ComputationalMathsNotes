\chapter[Finite differences in 1D]{Finite differences for \\ 
stationary problems in 1D}

\section{A model boundary-value problem}

\section{Second-order central difference}
\begin{equation}\label{eq: model 1D linear system}
\frac{1}{\Delta x^2}\begin{bmatrix}
 2&    -1&      &      &\\
-1&     2&    -1&      &\\
  &\ddots&\ddots&\ddots&\\
  &      &    -1&     2&-1\\
  &      &      &    -1& 2
\end{bmatrix}
\begin{bmatrix}U_1\\ U_2\\ \vdots\\ U_{P-2}\\ U_{P-1} \end{bmatrix}
=\begin{bmatrix}f_1\\ f_2\\ \vdots\\ f_{P-2}\\ f_{P-1} \end{bmatrix}
+\frac{1}{\Delta x^2}
\begin{bmatrix}\gamma_0\\ 0\\ \vdots\\ 0\\ \gamma_L \end{bmatrix}
\end{equation}


\section{Maximum principles}

\section{Well-posedness}

\section{Tridiagonal linear systems}
How can we solve a symmetric, tridiagonal linear system such as the one
\eqref{eq: model 1D linear system} arising from the finite difference 
scheme~\eqref{}?  To discuss this problem, consider a $5\times5$ matrix of the 
form
\[
\boldsymbol{A}=\begin{bmatrix}
\alpha_1& \beta_1&        &        &\\
 \beta_1&\alpha_1& \beta_2&        &\\
        & \beta_2&\alpha_3&\beta_3 &\\
        &        & \beta_3&\alpha_4&\beta_4\\
        &        &        & \beta_4&\alpha_5
\end{bmatrix}.
\]
A standard algorithm involves computing $5\times5$~matrices 
$\boldsymbol{L}$~and $\boldsymbol{D}$ of the form
\[
\boldsymbol{L}=\begin{bmatrix}
     1&      &      &      &\\
\ell_1&     1&      &      &\\
      &\ell_2&     1&      &\\
      &      &\ell_3&     1&\\
      &      &      &\ell_4&1
  \end{bmatrix}
\quad\text{and}\quad
\boldsymbol{D}=\begin{bmatrix}
d_1&   &   &   &\\
   &d_2&   &   &\\
   &   &d_3&   &\\
   &   &   &d_4&\\
   &   &   &   &d_5
  \end{bmatrix}
\]
having the property that
\[
\boldsymbol{A}=\boldsymbol{L}\boldsymbol{D}\boldsymbol{L}^T.
\]
Given a right-hand side vector~$\boldsymbol{b}$, we can solve the linear system
$\boldsymbol{A}\boldsymbol{x}=\boldsymbol{b}$ by solving in sequence the three
linear systems
\begin{equation}\label{eq: LDLT systems}
\boldsymbol{L}\boldsymbol{z}=\boldsymbol{b},\qquad
\boldsymbol{D}\boldsymbol{y}=\boldsymbol{z},\qquad
\boldsymbol{L}^T\boldsymbol{x}=\boldsymbol{y},
\end{equation}
because it will then follow that
\[
\boldsymbol{A}\boldsymbol{x}
    =\boldsymbol{L}\boldsymbol{D}\boldsymbol{L}^T\boldsymbol{x}
    =\boldsymbol{L}\boldsymbol{D}\boldsymbol{y}
    =\boldsymbol{L}\boldsymbol{z}=\boldsymbol{b}.
\]
Since $\boldsymbol{L}$ is lower triangular and $\boldsymbol{D}$ is diagonal,
we can easily compute $\boldsymbol{z}$, $\boldsymbol{y}$ and finally 
$\boldsymbol{x}$.  To see how, we write out the equations in the $5\times5$ 
case:
\begin{align*}
      z_1    &=b_1,& d_1y_1&=z_1,& z_1+\ell_1 z_2&=y_1,\\
\ell_1z_1+z_2&=b_2,& d_2y_2&=z_2,& z_2+\ell_2 z_3&=y_2,\\
\ell_2z_2+z_3&=b_3,& d_3y_3&=z_3,& z_3+\ell_3 z_4&=y_3,\\
\ell_3z_3+z_4&=b_4,& d_4y_4&=z_4,& z_4+\ell_4 z_5&=y_4,\\
\ell_4z_4+z_5&=b_5,& d_5y_5&=z_5,& z_5           &=y_5.
\end{align*}
Thus,
\begin{align*}
z_1&=b_1,           & y_1&=z_1/d_1,& z_5&=y_5,\\
z_2&=b_2-\ell_1 z_1,& y_2&=z_2/d_2,& z_4&=y_4-\ell_4z_5,\\
z_3&=b_3-\ell_2 z_2,& y_3&=z_3/d_3,& z_3&=y_3-\ell_3z_4,\\
z_4&=b_4-\ell_3 z_3,& y_4&=z_4/d_4,& z_2&=y_2-\ell_2z_3,\\
z_5&=b_5-\ell_4 z_4,& y_5&=z_5/d_5,& z_1&=y_1-\ell_1z_2.\\
\end{align*}





\section{General two-point boundary-value problem}

\section{An error bound}
