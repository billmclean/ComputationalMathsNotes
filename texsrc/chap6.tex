\chapter{Finite elements in 2D}

In Chapter~\ref{chap: FEM 1d} we considered a formally self-adjoint, linear, 
second-order differential operator~\eqref{eq: L self-adjoint}.  The 2D 
equivalent has the form
\begin{equation}\label{eq: L self-adjoint 2d}
\begin{aligned}
\mathcal{L}u&=-\nabla\cdot\bigl(a\nabla u\bigr)+cu\\
	&=-\frac{\partial}{\partial x}\biggl(a\,\frac{\partial u}{\partial x}\biggr)
	-\frac{\partial}{\partial y}\biggl(a\,\frac{\partial u}{\partial y}\biggr)
	+cu,
\end{aligned}
\end{equation}
where the coefficients $a$~and $c$ must be smooth functions of $x$~and $y$, and 
there is a constant~$a_{\min}$ such that
\[
a(x,y)\ge a_{\min}>0\quad\text{for $(x,y)\in\Omega$,}
\]
ensuring that $\mathcal{L}$ is \emph{uniformly elliptic}.  Here, as in 
Chapter~\ref{chap: finite diff 2d}, $\Omega$ is a bounded open subset 
of~$\mathbb{R}^2$ with a piecewise smooth boundary~$\Gamma=\partial\Omega$, but 
we will now suppose that
\[
\Gamma=\Gamma_{\mathrm{D}}\cup\Gamma_{\mathrm{N}},
\]
where $\Gamma_{\mathrm{D}}$~and $\Gamma_{\mathrm{N}}$ are non-overlapping, 
relatively closed subsets of~$\partial\Omega$ consisting of finitely many 
smooth curves.  (Thus, the intersection 
$\Gamma_{\mathrm{D}}\cap\Gamma_{\mathrm{N}}$ consists of finitely many
\emph{collision points}.) Our aim is to use the finite element method to compute 
numerical solutions to a \emph{mixed boundary-value problem} of the form
\begin{equation}\label{eq: self-adjoint bvp 2d}
\begin{aligned}
\mathcal{L}u&=f&&\text{in~$\Omega$,}\\
u&=g_{\mathrm{D}}&&\text{on~$\Gamma_{\mathrm{D}}$,}\\
a\,\frac{\partial u}{\partial n}&=g_{\mathrm{N}}&&
	\text{on~$\Gamma_{\mathrm{N}}$.}
\end{aligned}
\end{equation}
Here, $\partial u/\partial n$ is the derivative of~$u$ in the direction of the 
\emph{outward unit normal}~$\boldsymbol{n}$ for~$\Omega$, that is,
\[
\frac{\partial u}{\partial n}(x,y)
	=\boldsymbol{n}(x,y)\cdot\nabla u(x,y)
	\quad\text{for $(x,y)\in\partial\Omega$.}
\]
We refer to~$\Gamma_{\mathrm{D}}$~and $\Gamma_{\mathrm{N}}$ as the 
\emph{Dirichlet}~and \emph{Neumann} parts of the boundary, respectively, since 
we specify a Dirichlet boundary condition~$u=g_{\mathrm{D}}$ 
on~$\Gamma_{\mathrm{D}}$ and a Neumann boundary 
condition~$a\partial u/\partial n=g_{\mathrm{N}}$ on~$\Gamma_{\mathrm{N}}$. 

In the special case of a \emph{pure Dirichlet problem}, the Neumann part of the 
boundary is empty and so $u$ is specified on the whole of~$\Gamma$ (as 
in Chapter~\ref{chap: finite diff 2d}).  In the opposite case of a 
\emph{pure Neumann problem}, the Dirichlet part of the boundary is empty and so 
$a\,\partial u/\partial n$ is specified on the whole of~$\Gamma$.

\section{First Green identity}

Recall the \emph{divergence theorem} from vector calculus; on the right-hand 
side, the integral over~$\Gamma$ is with respect to \

\begin{theorem}\label{thm: divergence}
If the vector field $\boldsymbol{F}:\Omega\cup\Gamma\to\mathbb{R}^2$ is $C^1$, 
then
\[
\int_\Omega\nabla\cdot\boldsymbol{F}
	=\int_\Gamma\boldsymbol{F}\cdot\boldsymbol{n}.
\]
\end{theorem}

Written out more explicitly, if 
\[
\boldsymbol{F}(x,y)=P(x,y)\,\boldsymbol{i}+Q(x,y)\,\boldsymbol{j}
\quad\text{and}\quad
\boldsymbol{n}=n_x\,\boldsymbol{i}+n_y\,\boldsymbol{j},
\]
then the divergence theorem says that
\[
\iint_\Omega\biggl(\frac{\partial P}{\partial x}+\frac{\partial Q}{\partial y}
	\biggr)\,dx\,dy=\int_\Gamma\bigl(P\,n_x+Q\,n_y)\,ds,
\]
where $ds$ is the element of arc length along~$\Gamma$.  We also recall the 
following vector field identity.

\begin{lemma}\label{lem: div phi F}
For a $C^1$ scalar field~$\phi$ and a $C^1$ vector field~$\boldsymbol{F}$,
\[
\nabla\cdot(\phi\boldsymbol{F})=(\nabla\phi)\cdot\boldsymbol{F}
	+\phi\,\nabla\cdot\boldsymbol{F}.
\]
\end{lemma}

Together, Theorem~\ref{thm: divergence}~and Lemma~\ref{lem: div phi F} may be 
used to prove a 2D version of~\eqref{eq: int by parts}.

\begin{theorem}[First Green Identity]\label{thm: first Green}
If $u:\Omega\cup\Gamma\to\mathbb{R}$ is $C^2$, and if 
$v:\Omega\cup\Gamma\to\mathbb{R}$ is $C^1$, then
\[
\int_\Omega(\mathcal{L}u)\,v
	=\int_\Omega\bigl(a\nabla u\cdot\nabla v+cuv\bigr)
	-\int_\Gamma a\,\frac{\partial u}{\partial n}\,v.
\]
\end{theorem}
\begin{proof}
Taking $\phi=v$~and $\boldsymbol{F}=a\nabla u$ in Lemma~\ref{lem: div phi F}, 
we have
\[
\nabla\cdot(va\nabla u)=(\nabla v)\cdot(a\nabla u)+v\nabla\cdot(a\nabla u),
\]
so
\begin{align*}
(\mathcal{L}u)v&=\bigl(-\nabla\cdot(a\nabla u)+cu\bigr)v
	=-v\nabla\cdot(a\nabla u)+cuv\\
	&=(\nabla v)\cdot(a\nabla u)-\nabla\cdot(va\nabla u)+cuv
	=\bigl(a\nabla u\cdot\nabla v+cuv)-\nabla\cdot(va\nabla u).
\end{align*}
Applying Theorem~\ref{thm: divergence} with~$\boldsymbol{F}=va\nabla u$, it 
follows that
\[
\int_\Omega(\mathcal{L}u)v=\int_\Omega\bigl(a\nabla u\cdot\nabla v+cuv\bigr)
	-\int_\Gamma\boldsymbol(va\nabla u)\cdot\boldsymbol{n},
\]
which gives the desired identity because 
$(\nabla u)\cdot\boldsymbol{n}=\partial u/\partial n$.
\end{proof}

Since 
\[
\int_\Gamma a\,\frac{\partial u}{\partial n}\,v
	=\int_{\Gamma_{\mathrm{D}}} a\,\frac{\partial u}{\partial n}\,v
	+\int_{\Gamma_{\mathrm{N}}} a\,\frac{\partial u}{\partial n}\,v,
\]
we see that if $u$ is a $C^2$ solution of~\eqref{eq: self-adjoint bvp 2d}, and 
if $v$ is $C^1$, then
\begin{equation}\label{eq: Lu=f weak 2d}
\int_\Omega\bigl(a\nabla u\cdot\nabla v+cuv\bigr)=\int_\Omega fv
	-\int_{\Gamma_{\mathrm{N}}}g_{\mathrm{N}}v
	\quad\text{provided $v=0$ on $\Gamma_{\mathrm{D}}$.}
\end{equation}
Compare this property with its 1D equivalent~\eqref{eq: Lu=f weak 1d}.

\section{Triangulation and nodal basis}

\begin{figure}
\caption{A regular triangulation.}\label{fig: good Th}
\begin{center}
\includegraphics[scale=0.7]{../src/chap6/good_triangulation.pdf} 
\end{center}
\end{figure}

\begin{figure}
\caption{This triangulation fails to be regular.}\label{fig: bad Th}
\begin{center}
\includegraphics[scale=0.7]{../src/chap6/bad_triangulation.pdf} 
\end{center}
\end{figure}

Assume now that $\Omega$ is a polygon.  It follows by induction on the number 
of vertices that there exists a \emph{triangulation} of~$\Omega$, that is, a 
finite set~$\mathcal{T}$ of \emph{non-overlapping} closed triangles whose 
union is the closure of~$\Omega$. A triangulation~$\mathcal{T}$ is 
\emph{regular} if the following two conditions are satisfied:
\begin{enumerate}
\item No triangle in~$\mathcal{T}$ is \emph{degenerate}, that is, 
no~$K\in\mathcal{T}$ has collinear vertices.
\item The intersection $K_1\cap K_2$ of any two distinct triangles $K_1$, 
$K_2\in\mathcal{T}$ is either empty, a common edge or a common vertex.
\end{enumerate}
For example, Figure~\ref{fig: good Th} shows a regular triangulation with 
7~vertices, or \emph{nodes}, numbered in red, and 6~triangles, or 
\emph{elements}, numbered in blue.  However, the triangulation in 
Figure~\ref{fig: bad Th} is not regular: the intersection of triangles 2~and 6 
is an edge of triangle~6 but is not (the whole of) an edge of triangle~2.
Vertex~7 in Figure~\ref{fig: bad Th} is said to be a \emph{hanging node}.

Denote the maximum element diameter 
by~$h=\max_{K\in\mathcal{T}}\operatorname{diam}(K)$, and let $V_h$ denote the 
vector space consisting of those functions~$v:\Omega\cup\Gamma\to\mathbb{R}$ 
that are continuous and piecewise-linear with respect to~$\mathcal{T}$.  Thus,
if $v\in V_h$ then for each~$K\in\mathcal{T}$ there are coefficients 
$c^\brak{K}_0$, $c^\brak{K}_1$~and $c^\brak{K}_2$ such that
\begin{equation}\label{eq: v K 1 x y}
v(x,y)=c^\brak{K}_0+c^\brak{K}_1x+c^\brak{K}_2y
	\quad\text{for $(x,y)\in K$.}
\end{equation}
Let $\mathsf{n}^\brak{K}_1$, $\mathsf{n}^\brak{K}_2$, $\mathsf{n}^\brak{K}_3$ 
denote the vertices of the triangle~$K$, and let $\psi^\brak{K}_1$, 
$\psi^\brak{K}_2$, $\psi^\brak{K}_3$ denote the unique linear functions 
satisfying
\begin{equation}\label{eq: psi node triangle}
\psi^\brak{K}_p(\mathsf{n}^\brak{K}_p)=\delta_{p,q}
	\quad\text{for $p$, $q\in\{1,2,3\}$.}
\end{equation}
In Section~\ref{sec: barycentric}, we will derive explicit representations of 
these functions.  The property~\eqref{eq: psi node triangle} implies that if 
$v\in V_h$ then
\[
v(x,y)=\sum_{q=1}^3v(\mathsf{n}^\brak{K}_q)\psi^\brak{K}_q(x,y)
	\quad\text{for $(x,y)\in K$,}
\]
showing that $v$ is uniquely determined by its values at the nodes 
of~$\mathcal{T}$.

\begin{figure}
\caption{A piecewise-linear ``tent function'', equal to~$1$ at one node, and 
$0$ at all other nodes.}\label{fig: tent func}
\begin{center}
\includegraphics[scale=0.6]{../src/chap6/tent_func.pdf}
\end{center}
\end{figure}

In fact, suppose that $\mathsf{n}_1$, $\mathsf{n}_2$, \dots, $\mathsf{n}_N$ is 
an enumeration of the nodes of~$\mathcal{T}$.  For~$1\le j\le N$, we 
define~$\chi_j\in V_h$ by requiring
\[
\chi_k(x_j)=\delta_{jk}\quad\text{for $j$, $k\in\{1, 2, \dots, N\}$.}
\]
Figure~\ref{fig: tent func} shows an example of such a ``tent function''.  
If $v\in V_h$, then
\[
v(x,y)=\sum_{k=1}^N v(\mathsf{n}_k)\chi_k(x,y)
	\quad\text{for $(x,y)\in\Omega\cup\Gamma$,}
\]
and we call $\{\chi_1,\chi_2,\ldots,\chi_N\}$ the \emph{nodal basis} for~$V_h$
(Exercise~\ref{ex: nodal basis}).

\section{Finite element method}

Suppose that a regular triangulation~$\mathcal{T}$ is \emph{aligned} with 
the decomposition $\Gamma=\Gamma_{\mathrm{D}}\cup\Gamma_{\mathrm{N}}$ of the 
boundary of~$\Omega$.  This assumption means that $\Gamma_{\mathrm{D}}$ 
is a union of edges of triangles in~$\mathcal{T}$ (in which 
case, the same must be true of~$\Gamma_{\mathrm{N}}$).  The vertices lying on 
the Dirichlet boundary~$\Gamma_{\mathrm{D}}$ are called the \emph{fixed nodes}, 
because the values of the solution~$u$ are fixed at these points.  The 
remaining vertices are called the \emph{free nodes}; these belong 
to~$\Omega\cup\Gamma_{\mathrm{N}}$, but note that the collision points, where
$\Gamma_{\mathrm{D}}$~and $\Gamma_{\mathrm{N}}$ meet, are among the fixed nodes.

Suppose that there are $M$~free nodes and $R$~fixed nodes. It is convenient to 
number the nodes so that free nodes come first, followed by the fixed nodes.
That is, $\mathsf{n}_1$, $\mathsf{n}_2$, \dots, 
$\mathsf{n}_M$ are free, and $\mathsf{n}_{M+1}$, $\mathsf{n}_{M+2}$, \dots, 
$\mathsf{n}_{M+R}$ are fixed.  

Let $g_{\mathrm{D},h}:\Gamma_{\mathrm{D}}\to\mathbb{R}$ be a piecewise-linear 
approximation to~$g_{\mathrm{D}}$.  An obvious choice is the interpolant, so
that
\[
g_{\mathrm{D},h}(\mathsf{n}_k)=g(\mathsf{n}_k)\quad\text{for $M+1\le k\le M+R$.}
\]
We define the trial set
\[
S_h=\{\,v\in V_h:\text{$v=g_{\mathrm{D},h}$ on $\Gamma_{\mathrm{D}}$}\,\}
\]
and the test space
\[
S_h=\{\,v\in V_h:\text{$v=0$ on $\Gamma_{\mathrm{D}}$}\,\}.
\]
Recalling \eqref{eq: Lu=f weak 2d}, the finite element 
solution~$u_h\in S_h$ is then defined by requiring that
\begin{equation}\label{eq: FEM 2d}
\int_\Omega\bigl(a\nabla u_h\cdot\nabla v+cu_hv\bigr)=\int_\Omega fv
	-\int_{\Gamma_{\mathrm{N}}}g_{\mathrm{N}}v
	\quad\text{for all $v\in T_h$.}
\end{equation}

\section{Barycentric coordinates}\label{sec: barycentric}

\section{Error bounds}

\begin{Exercises}

\exercise\label{ex: nodal basis}
Prove that the functions~\eqref{eq: chi 2d} form a basis for the 
piecewise-linear, finite element space~$V_h$, that is, prove that the nodal 
basis really is a basis. 
\end{Exercises}
