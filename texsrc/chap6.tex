\chapter{Finite elements in 2D}

In Chapter~\ref{chap: FEM 1d} we considered a formally self-adjoint, linear, 
second-order differential operator~\eqref{eq: L self-adjoint}.  The 2D 
equivalent has the form
\begin{equation}\label{eq: L self-adjoint 2d}
\begin{aligned}
\mathcal{L}u&=-\nabla\cdot\bigl(a\nabla u\bigr)+cu\\
	&=-\frac{\partial}{\partial x}\biggl(a\,\frac{\partial u}{\partial x}\biggr)
	-\frac{\partial}{\partial y}\biggl(a\,\frac{\partial u}{\partial y}\biggr)
	+cu,
\end{aligned}
\end{equation}
where the coefficients $a$~and $c$ must be smooth functions of $x$~and $y$, and 
there is a constant~$a_{\min}$ such that
\[
a(x,y)\ge a_{\min}>0\quad\text{for $(x,y)\in\Omega$,}
\]
ensuring that $\mathcal{L}$ is \emph{uniformly elliptic}.  Here, as in 
Chapter~\ref{chap: finite diff 2d}, $\Omega$ is a bounded open subset 
of~$\mathbb{R}^2$ with a piecewise smooth boundary~$\Gamma=\partial\Omega$, but 
we will now suppose that
\[
\Gamma=\overline{\Gamma_{\mathrm{D}}\cup\Gamma_{\mathrm{N}}},
\]
where $\Gamma_{\mathrm{D}}$~and $\Gamma_{\mathrm{N}}$ are disjoint subsets 
of~$\partial\Omega$ consisting of finitely many smooth curves.  Our aim is to 
use the finite element method to compute numerical solutions to a \emph{mixed 
boundary-value problem} of the form
\begin{equation}\label{eq: self-adjoint bvp 2d}
\begin{aligned}
\mathcal{L}u&=f&&\text{in~$\Omega$,}\\
u&=g_{\mathrm{D}}&&\text{on~$\Gamma_{\mathrm{D}}$,}\\
a\,\frac{\partial u}{\partial n}&=g_{\mathrm{N}}&&
	\text{on~$\Gamma_{\mathrm{N}}$.}
\end{aligned}
\end{equation}
Here, $\partial u/\partial n$ is the derivative of~$u$ in the direction of the 
\emph{outward unit normal}~$\boldsymbol{n}$ for~$\Omega$, that is,
\[
\frac{\partial u}{\partial n}(x,y)
	=\boldsymbol{n}(x,y)\cdot\nabla u(x,y)
	\quad\text{for $(x,y)\in\partial\Omega$.}
\]
We refer to~$\Gamma_{\mathrm{D}}$~and $\Gamma_{\mathrm{N}}$ as the 
\emph{Dirichlet}~and \emph{Neumann} parts of the boundary, respectively, since 
we specify a Dirichlet boundary condition~$u=g_{\mathrm{D}}$ 
on~$\Gamma_{\mathrm{D}}$ and a Neumann boundary 
condition~$a\partial u/\partial n=g_{\mathrm{N}}$ on~$\Gamma_{\mathrm{N}}$. 

In the special case of a \emph{pure Dirichlet problem}, the Neumann part of the 
boundary is empty and so $u$ is specified on the whole of~$\Gamma$ (as 
in Chapter~\ref{chap: finite diff 2d}).  In the opposite case of a 
\emph{pure Neumann problem}, the Dirichlet part of the boundary is empty and so 
$a\,\partial u/\partial n$ is specified on the whole of~$\Gamma$.

\section{First Green identity}

Recall the \emph{divergence theorem} from vector calculus; on the right-hand 
side, the integral over~$\Gamma$ is with respect to \

\begin{theorem}\label{thm: divergence}
If the vector field $\boldsymbol{F}:\Omega\cup\Gamma\to\mathbb{R}^2$ is $C^1$, 
then
\[
\int_\Omega\nabla\cdot\boldsymbol{F}
	=\int_\Gamma\boldsymbol{F}\cdot\boldsymbol{n}.
\]
\end{theorem}

Written out more explicitly, if 
\[
\boldsymbol{F}(x,y)=P(x,y)\,\boldsymbol{i}+Q(x,y)\,\boldsymbol{j}
\quad\text{and}\quad
\boldsymbol{n}=n_x\,\boldsymbol{i}+n_y\,\boldsymbol{j},
\]
then the divergence theorem says that
\[
\iint_\Omega\biggl(\frac{\partial P}{\partial x}+\frac{\partial Q}{\partial y}
	\biggr)\,dx\,dy=\int_\Gamma\bigl(P\,n_x+Q\,n_y)\,ds,
\]
where $ds$ is the element of arc length along~$\Gamma$.  We also recall the 
following vector field identity.

\begin{lemma}\label{lem: div phi F}
For a $C^1$ scalar field~$\phi$ and a $C^1$ vector field~$\boldsymbol{F}$,
\[
\nabla\cdot(\phi\boldsymbol{F})=(\nabla\phi)\cdot\boldsymbol{F}
	+\phi\,\nabla\cdot\boldsymbol{F}.
\]
\end{lemma}

Together, Theorem~\ref{thm: divergence}~and Lemma~\ref{lem: div phi F} may be 
used to prove a 2D version of~\eqref{eq: int by parts}.

\begin{theorem}[First Green Identity]\label{thm: first Green}
If $u:\Omega\cup\Gamma\to\mathbb{R}$ is $C^2$, and if 
$v:\Omega\cup\Gamma\to\mathbb{R}$ is $C^1$, then
\[
\int_\Omega(\mathcal{L}u)\,v
	=\int_\Omega\bigl(a\nabla u\cdot\nabla v+cuv\bigr)
	-\int_\Gamma a\,\frac{\partial u}{\partial n}\,v.
\]
\end{theorem}
\begin{proof}
Taking $\phi=v$~and $\boldsymbol{F}=a\nabla u$ in Lemma~\ref{lem: div phi F}, 
we have
\[
\nabla\cdot(va\nabla u)=(\nabla v)\cdot(a\nabla u)+v\nabla\cdot(a\nabla u),
\]
so
\begin{align*}
(\mathcal{L}u)v&=\bigl(-\nabla\cdot(a\nabla u)+cu\bigr)v
	=-v\nabla\cdot(a\nabla u)+cuv\\
	&=(\nabla v)\cdot(a\nabla u)-\nabla\cdot(va\nabla u)+cuv
	=\bigl(a\nabla u\cdot\nabla v+cuv)-\nabla\cdot(va\nabla u).
\end{align*}
Applying Theorem~\ref{thm: divergence} with~$\boldsymbol{F}=va\nabla u$, it 
follows that
\[
\int_\Omega(\mathcal{L}u)v=\int_\Omega\bigl(a\nabla u\cdot\nabla v+cuv\bigr)
	-\int_\Gamma\boldsymbol(va\nabla u)\cdot\boldsymbol{n},
\]
which gives the desired identity because 
$(\nabla u)\cdot\boldsymbol{n}=\partial u/\partial n$.
\end{proof}

Since 
\[
\int_\Gamma a\,\frac{\partial u}{\partial n}\,v
	=\int_{\Gamma_{\mathrm{D}}} a\,\frac{\partial u}{\partial n}\,v
	+\int_{\Gamma_{\mathrm{N}}} a\,\frac{\partial u}{\partial n}\,v,
\]
we see that if $u$ is a $C^2$ solution of~\eqref{eq: self-adjoint bvp 2d}, and 
if $v$ is $C^1$, then
\begin{equation}\label{eq: Lu=f weak 2d}
\int_\Omega\bigl(a\nabla u\cdot\nabla v+cuv\bigr)=\int_\Omega fv
	-\int_{\Gamma_{\mathrm{N}}}g_{\mathrm{N}}v
	\quad\text{provided $v=0$ on $\Gamma_{\mathrm{D}}$.}
\end{equation}
Compare this property with its 1D equivalent~\eqref{eq: Lu=f weak 1d}.

\section{Triangulation}

\section{Matrix assembly algorithm}

\section{Barycentric coordinates}

\section{Error bounds}
